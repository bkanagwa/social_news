\section{Conlusion} % (fold)
\label{sec:conlusion}

In this paper, we used $\pi-calculus$ and P/T nets to specify and analyze a social networking application, {\tt Social News}. Our specifications showed two levels of interactions: within a group with
secrecy and confidentiality and at a global level. As for the analysis, we focused on verifying the correctness of the interactions and the achievability of the overall goal of the application.

In introduction to this paper, we claimed that one should turn to algebraic approaches instead of the purely statistical ones, in order to gain a deeper understanding of behaviors in social networks.
However, our claim does not suggest to squarely rule out the statistical approaches. Rather, we advocate for a hybrid approach, given the wealth of data/information available in social networks.
Indeed, the algebraic approaches could provide a concise and accurate model of the social networking application at hand, while the statistical approaches help identify new trends or development of the
initial model for further refinement. In short, our future research will seek to reconcile these two types of approaches.

By attempting to analyze a social networking using algebraic approaches, this research aims at addressing more complex questions such as \emph{how does the behavior of a social network evolve over
time?} Such understanding can help further automate processes within a social network and reach the ideal situation where humans interact with computers in a transparent and flexible manner.

In this paper, we have adopted $CTL^*$ as a formalism to represent a task being carried out. Then we represented the execution of the task using P/T net on top of $\pi-calculus$ terms. In the future,
we wish to investigate how to model check the execution of the task using the P/T net. More generally, following the foundation laid by Groote and Reniers (see~\cite{Groote-Reniers:01}), we wish to
investigate more appropriate methodologies for process algebraic verification in social networking applications.

Another important question raised in this research is \emph{task division}. Given a task submitted to a community in a social network, how to come up with a possibly automated fair division of the task
into subtasks among its members.

% section conlusion (end)