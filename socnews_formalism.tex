\section{Formalisms} % (fold)
\label{sec:formalisms}

\textbf{Need to rework this section}

\subsection{$\pi-Calculus$} % (fold)
\label{sub:_pi_calculus_}

The $\pi-calculus$ is a process algebra which supports the modeling of mobile processes. It is based upon the notion of
naming and term rewriting. Its simplest and original form has been presented in~\cite{Milner-Parrow-Walker:89}. There
have been several extensions to the $\pi-calculus$: \emph{polyadic} (where prefixes are no longer primitive),
\emph{typed} (introducing sorts to differentiate the types of channels in use and the values transmitted across those
channels), etc. In~\cite{Cardelli-Ghelli-Gordon:05}, Cardelli et al introduced a variant of $\pi-calculus$ that supports
group creation and secrecy. Its syntax is based on an asynchronous choice free polyadic $\pi-calculus$. Since
confidentiality is a key security requirement in our application, private group creation and secrecy are therefore
central to our application. Thus, we adopt this variant of the $\pi-calculus$. In this section we overview the
syntax\footnote{The reader will notice a slight difference with the $\pi-calculus$ presented
in~\cite{Cardelli-Ghelli-Gordon:05}. We reintroduced the internal prefix and non deterministic choice operator in the
syntax, since they do not change the typing rules and do not impact on the secrecy mechanism.}, semantics and type
system.

In the variant of the $\pi-calculus$ we adopted, the types specify, for each channel, its group ant the types of values
that can be exchanged on that channel.

\begin{displaymath}
	T::=G[T_1,\ldots, T_n]
\end{displaymath}

Let $\mathbb{C}=\{a,b,c,\ldots\}$ be a countably infinite set of channels. A process, the main element of the syntax of the language, is defined as follows. 
\[
P::= \left. \begin{array}{l l}
a(b_1:T_1,\ldots,b_k:T_k).P & \quad \mbox{channel input}\\
\overset{-}{a}<b_1,\ldots,b_k> & \quad \mbox{channel output}\\
\tau.P & \quad \mbox{internal prefix}\\
(\nu G)P & \quad \mbox{group creation}\\
(\nu a:T)P & \quad \mbox{restriction}\\
P\mid Q & \quad \mbox{parallel composition}\\
P + Q & \quad \mbox{choice}\\
!P & \quad \mbox{replication}\\
\mathbf{0} & \quad \mbox{inactivity}\\
\end{array} \right.
\]

Apart from the usual garbage collection rules ($\mathbf{0}\equiv (\nu a:T)\mathbf{0}$ and $a\notin fn(P)\implies (\nu
a:T)P\equiv P$), the operational semantics of the language is standard (see~\cite{Parrow:01}). New rules are however
introduced to support group creation. First, we discuss the new structural congruence rules.
\[
\left.\! \begin{array}{l}
P\equiv Q\implies (\nu G)P\equiv (\nu G)Q \\
(\nu G_1)(\nu G_2)P\equiv (\nu G_2)(\nu G_1)P\\
G\notin fg(T)\implies (\nu G)(\nu a:T)P\equiv (\nu a:T)(\nu G)P\\
G\notin fg(P)\implies (\nu G)(P\mid Q)\equiv P\mid (\nu G)Q
\end{array} \right.\!
\]

The only new reduction rule comes as follows.

\[P\rightarrow Q\implies (\nu G)P\rightarrow(\nu G)Q\]

Finally, we briefly discuss the typing rules, especially those related to groups and type channels.
\[
\frac{E\vdash\diamond \quad G\notin dom(E)}{E,G\vdash\diamond}\quad
\frac{E,G\vdash P}{E\vdash (\nu G)P}
\]

\[\frac{G\in dom(E) \quad E\vdash T_1\quad\cdots\quad E\vdash T_n}{E\vdash G[T_1,\cdots, T_n]}\]

\[
\frac{E\vdash a:G[T_1,\ldots,T_n] \quad E,b_1:T_1,\ldots,b_n:T_n\vdash P}{E\vdash a(b_1:T_1,\ldots,b_n:T_n).P}
\]

\[
\frac{E\vdash a:G[T_1,\ldots,T_n] \quad E\vdash b_1:T_1\quad\cdots\quad E\vdash b_n:T_n}{E\vdash \overset{-}{a}<b_1,\ldots,b_n>}
\]

% subsection _pi_calculus_ (end)

\subsection{Petri Nets} % (fold)
\label{sub:petri_nets}

\emph{Petri Nets} are a well known mathematical formalism to specify and analyze concurrent systems. Unlike
$\pi-calculus$, they are automata-based formalisms and therefore, offer a graphical view of the causality and the
concurrency exhibited by a system. There are several variants of Petri nets: \emph{elementary nets}, \emph{place and
transition nets} $(P/T \mbox{ nets})$, \emph{colored Petri nets} $(CPN)$, \emph{object nets}, etc. In this paper, we use
the P/T nets formalism, and therefore only overview this variant in this paper. A reader interested in elaborate
knowledge in Petri nets in general is referred to Reisig's introductory papers
(see~\cite{Reisig:85,Reisig:98a,Reisig:98b}).

Formally, a $P/T \mbox{ net}$ is a tuple $N=<P,T,F,W,M_0>$, where $P$ is a finite set of places (also known as
conditions), $T$ is a finite set of transitions (also known as actions), $F\subseteq (P\times T)\cup (T\times P)$ is the
flow relation, $W: F\rightarrow (\mathbb{N}\setminus\{0\})$ is the arc weight mapping and $M_0: P\rightarrow \mathbb{N}$
is the initial marking. Let $x\in P\cup T$, the set $^\bullet x=\{x^\prime, (x^\prime,x)\in F\}$ is called the pre-set of
$x$, while the set $x^\bullet=\{x^{\prime\prime}, (x, x^{\prime\prime})\in F\}$ is called the post-set of $x$.

Given a $P/T$ net $N=<P,T,F,W,M_0>$ and $M: P\rightarrow\mathbb{N}$ one of its markings, a transition $t~(t\in T)$ is
$M-$\emph{enabled} ($M\overset{t}{\rightarrow}$) iff $\forall p\in~ ^\bullet p: M(p)\geq W(p,t)$. When an
$M-$\emph{enabled} transition $t$ fires and produces a successor marking $M^\prime$ (notation being
$M\overset{t}{\rightarrow}M^\prime$) we have $\forall p\in P, M^\prime(p)=M(p)-\hat{W}(p,t)+\hat{W}(t,p)$ where
$\hat{W}(p,t)=W(p,t)$ if $(p,t)\in F$ and $0$ otherwise.

Modeling a concurrent system using a $P/T$ net (or any other variant of \emph{Petri Net}) makes it easier not only to
visualize the causality and concurrency, but also to analyze its key properties. These include checking the reachability
of a marking, finding a path to a marking (from $M_0$), checking whether a transition is enabled, checking deadlock
markings and the boundedness of the net. Most of these properties rely on the \emph{reachability} of a marking. Checking
the reachability of a marking consists of building the reachability graph and locating the marking within. However, when
the net is not $k-safe$, i.e., there cannot be more than $k$ (any arbitrary number) tokens at any place of the net during
its execution, \emph{coverability} (an approximation of the reachability) is used instead. Finally, when the reachability
graph is huge, \emph{structural analysis} is resorted to in order to verify properties.

% subsection petri_nets (end)

% section formalisms (end)