%% ------------------------------------------------------------ 
%% File: socnews.tex
%% Author: Jose Ghislain Quenum & Benjamin Kanagwa
%% Created: 10 2009
%% Updated: 11 2009
%% ------------------------------------------------------------

\documentclass[acmjacm]{acmtrans2m}

\usepackage{alltt}
\usepackage{amssymb}
\usepackage{amsmath}
\usepackage{amsfonts}
\usepackage{array}
\usepackage{booktabs}
\usepackage{boxedminipage}
\usepackage{fancybox}
\usepackage{fancyvrb}
\usepackage{float}
\usepackage{graphicx}
\usepackage{hhline}
\usepackage[latin1]{inputenc}
\usepackage{latexsym}
\usepackage{lineno}
\usepackage{listings}
\usepackage{paralist}
\usepackage{relsize}
\usepackage{subfig}
\usepackage{tabularx}
\usepackage{theorem}
\usepackage{url}
\usepackage{xcolor}

\graphicspath{{images/}}
\sloppy

\newtheorem{proposition}{Proposition}[subsection]

\markboth{J. G. QUENUM and B. KANAGWA}{``Understand! Do Not just Guess!''}

% \markboth{J. G. QUENUM and B. KANAGWA}{``A Process View of Collaboration in Social Networks''}

% \title{``Understand! Do Not just Guess!''\\ A Process View of Collaboration in Social Networks}

\title{A Process View of Collaboration in Social Networks}

\author{ Jos\'{e}~Ghislain~QUENUM\\Makerere University \and Benjamin KANAGWA\\Makerere University}

\begin{abstract}

This paper proposes an algebraic approach to specify and analyze collaboration in a social network. We represent members
of the social network and their behaviors as processes. Using a social networking application, {\tt Social News}, we
illustrate how to specify and analyze the behaviors of members of a social network using both $\pi-calculus$ and
\emph{P/T net}, a well known variant of \emph{Petri Net}. In combining both formalisms we wish to offer both the
expressivity of term-rewriting formalisms (here, $\pi-calculus$) and the clarity of automata-based ones (here, \emph{P/T
net}). We used $\pi-calculus$ for intra-groups activities (group formation and group level decision making), while
\emph{P/T net} depicts the inter-group activities. In our specifications we took a special care of secrecy and
confidentiality.

\end{abstract}

\category{C.2.4}{Computer Communication Networks}{Distributed Systems} 
\category{F.1.2}{Computation by Abstract Devices}{Modes of Computation} 
\terms{Specifications, Theory, Verification} 
\keywords{Social network, crowd casting, complex adaptive systems, crowd sourcing, formal specifications, $\pi-calculus$, Petri nets}

\begin{document}
	
	\begin{bottomstuff}
	Authors' address: J. G. QUENUM \& B. KANAGWA, 
	Makerere University,
	Plot 56 Makerere University Road, Kampala, Uganda
	P.O. Box 7062.\newline
	\{joque,bkanagwa\}@cit.mak.ac.ug
	\end{bottomstuff}
	
\maketitle

-----BEGIN PGP MESSAGE-----
Version: GnuPG v1.4.10 (Darwin)

hQIMA77xv53FwjH7AQ//VC49ghnH80fY2XiihICDUOdcOT4LWtnGzlPykTca2W2a
WKpKdvSudY9Z+yijSIg1TsnU5HQIRWloCmuHhyEQVR3VGFTZHk6v5BPczJCRcVtb
Z1H2frh14ykt2MIBJhAXF7xzdf/DCyLMwO7OnMJJaPLGhxLiUWeGLvv0aQymt+Ux
XYInxKohcHdGw3NoXhSZEcpWiHYPWKO6qtV7veDZ59kKk/S8tWLSR6UeAiiTn7JU
Fp1x582AfHHVjUMiiv4Ntbd2S4P+Tp8k3xbj+SpYxN1QglwhgSQnnNfyy9/eoiM/
v8+9ngN0tMkfeuh1tKLJqilhXwr8sMJhLK2Osvtqggh4C5ijJqg7Mt86NYWoS35a
q2VQCwALQqZDM6+duiGjnE56Ti81hpiAtCtLjZmv6NCk5LxZFGrmuOfIRKJe2egE
HK8u2FNj2QIeLpvnTwmgjJ8fbr0pmlJGe2LkQyYeDw7orqctiqlVnD4Olw66vDgo
y3HFuvG3i1H52OY7UsuqVhM98OAXrOr60HcQoY+P348KpUWwWTCY01A2CGZ3/jra
d4O+INx4cm4Tdu+rtPBWSFGpz841E5ss15Q/zzbj5VS6jr/ZFK1mWikN8sube+as
cZk5JXLWZy1O2HsjtoHBcfjJMrgFqsIF3aXKUTHRSPhDHrjgkfYsZTYSvyd5jYrS
7QGIsvcm8Hxl/pjZckEnkBdr3Uysnzl09yDw+R1iGq6TiXoCJKSJeP9Or5n5vRmu
0+n7KOOEe0ZQqBapTybQg2nYFmMKMlwIHC4JC3qA8W/OGXvkEViUwLVF20cmxL7z
RbxQGj1/E8SXm/XfNpSCkfrNy0rSySbWDwWnAv1Zyzr6sCydJQEZl+6eqUimj1QC
M/JYQvSU0NKSSPIBfr885rid4kooNMC88wDf5Q3Zj1Figxhadyv8mMfRe1eebv/P
PE+T1czaVb+JmDvT+NPuzkL0wLwToSe86QhCBTHxyCW/DEkJp8T/Ic5Sm63Ymy3o
m6LA8Bq5BlIGgEJngwlDH6loqdIoVwT/e94s8uXKFmyzFnkggnB6/nZX+VxfCKGy
YUWwcOdeOa41tCbCt+HX/c5YOE/jmgBdgbuROfXGg7Y4AUcVOw+MBEnbDLiXeKBt
m93dR0OzjqGF4xGWhpH/D/GVDOt/GrjYO14jmqx7KxagPXSy/KnsiNVdcB+13iJw
HHB6oPYkIOIX1hXlSt5/0knUpkIg4p4xaZOtvY0E4Hk3IiGo5XtXIEW50PRi8oSG
a1aI3XLra43KnVV47RQuKmJORL/myC2cyHp/dGmeVtfp79R2k/YKhu2R8kgoDM8A
jP+KpPLmu92UUvjtPhjRqAiEXhtKKUBLPk3d1QAYR6k2gNS2Q1FB+akTMkNqSsuz
5WRzaTOvf6zcIg1S7p4rQxI2SsBGXATXBsw0xHdJAeVl/qU5T3lNhlGkqv/JYdwz
OL6E0+UrKLNACk8eugF/pa0hkcxA3L2THVhVamVFUIEzW6OboZ0WnVXqwXsT/CDI
CNCcr9HR58KTjMfum0VgNhJwLkpT48J4OD9HqiR4g3E3XSgJpmFVbjfNnYgplAVf
/4Bo5yQCR/uRqB+HKudx786CKEToQ9U5s9eTxYKCTbYwXioNjZskdxRKFud/Sss+
BMHUuS+PW7yyVoQ+Lsb75ZZd352FXKcTVTwlPPtk5EHhBgEa2MIkrADBcZZeY+Kk
JcRyO73jp8A8cax19uSp7fNPkk+imi7j2mZ0/Scszaiw7mCHKsyeeVp6hXy2yMG7
E9HYnQsKN+sBq5tPtewgGTj4zdHAer4bSaQ9eLJmvucVMuJqzr6y9BkSjWGDraLa
odVyXi6hiAWv8vQg768F+tKvz8BZH+bJ9Bn2hWDqGoLPrlgEBs+pjxkMrhpX1uVt
ndJEe6LqKl2xNp/PZ/vNPKKP7E1/7AEjEda2D9e/VK233sJ/Gc/wYVMqw17im4z+
XYPBPxnIgVpnk5/jnNlzTdqg+Ru7KpkzwFiNPpFu+tr7HqM/WowG2poNB576CTVB
nBFOKmReqomwcd06wk5bzpsDQCtD9U2LnPuHPas64bvVt+4hcgU8vt3gAwcJmzFa
aIePFOmlivdbHZjQH+ofMjCUKpFbWr/eVeK+WWFmoTosLKupmUsJpyGLCfa5mMZJ
9QSXCnY578LkuufpOQW1gl2XrcM5UuXbtwUZOkbkb4t0Fixi+lB0HI103bTHlFaX
PznEwPCwkpjGJC7TUOu6KTUoa1msihiXuG0t9E6W3kCLtaYRriStG+Y+/PXSPrD4
QH4svcL3LrBiboyu0RErCnqF4adnRzQ40tYNsPZ6l/9E3cJIs/6dVZSxLPJX6TQm
F2DdGS/hRcdiigNd6OaWgYHT290mVZHkIpNexhmg9H0HCDGWHUy3iddzlbZ6UhFe
GrfXkPXuDMWBcfoz6iQmvUrPEQlquDrlhoS6b8xlQF8lcY7eVV5Jic8fzOiIA3Wk
DE872adxK/J3rXEpHOfN/s99/eVWTrxT8kJ88tLIw5ymqaJZ2pjK44ebpWldY2ub
BPG4CcLSWhYe6gpZmqm7KIEQxE00w2qMM1YcVUHpVnd9QX8Y0JTu2XrZLmiyYZgl
I/dO1dYHhd94qwRZvu2KjQm5N5arDp9/FNA/UHROTFTEkCdH8BDU7tCAmsBk+tdP
BEyWzr9AQpaMB2F0c4DIZXUOQdtSaj2ru9kb7gSYpQf0IIhUlvePVrbEE4YJ019Y
DzoiVPe+EJd+xiiZAER98mL7IBz/bracdYTFuvzpuEnJtcw7nvjBQeRRWD0Gvi4K
XqQH6zua513QEblHIrfvcPAw0RC+9c2/PFemQI2phSibTehh5HYFZnHJLw8EUMkF
ieT8lkRRmDat6gMmeSakixoDFIdZsS8WXAxKh4exATV8ygW5z3hNfg19ow58Rz0a
wl8ABm3TA/EniE8t2LW0Jz+fMwZ2kqc3HwjribdngBHHeVwWJyvDwzi7Y3qlEpuo
KTQCG9wlrdaiOWdWhIiJNZ4wOsZDl+kmAnfucZaEcKJFyZr4D6z3bK8AMG4CM83D
e1MPxJWRpsgprfp3xHVqr2ZXAMe52Qri7yoM4VubxtX+kEBS5Qv72zIdrGncWWAg
CHMstz3I9c+VeuR3RxE+UsLk6p29DV4Sh86Z5e4N5qPavUn6F0ljR4KftyxgV5Vl
IL//0VQKd5u6FUt4wV8eiMIhhmAfIZ5SQaHzqVQ/fBYkpGk4sRTpLMxtUSZEOJ51
n/aDedTVBACKFwq4qWIBeCNFYDslPmTe4feYu+13fIbBK7XxoJpS71L9w4qnNu/8
UlQ/iG+LP+MqVCtMVTH35vnSIlvij+XmqmrXETJQqzxIjTsgluYQz040bz123vr1
dSBIAvVVyQ46ySlD1qs2fF+vgtuEEZjatj0hS4hSxKbpNxF6vACKhGb+2QU9K9UA
P5b++soX22MidubFVO7FzX6yssKIo1dNCQOx6zgoy6lGVg9auZ2I0c5r49lPAnI4
nyMvFS3tJp5jSx/f8KQBOPO/9mheI8XttRH4sacP2KxFo2X6glusEHF+nnr1Pazv
1EKU0LwFT0LfYn8bBDOFnTJ5MeDA4bhzNcRgq83Hwb9MvH3j/g4qhdkDhhkgofJj
WPRCcf+ohrjuMs1c20MRoEqnt5SrpbhZskzAcHFrUpicrCpM+qLDFv/v+QHuxpOn
GQpb33wd9XKcd9pL/z/jhohZiiSnJWM+bPKFN0bQuSSdt+nyzuwJYWU5e6eLaD04
1goEP5clY7JGUOumnt80Hn5gg0B5Xs4mkOthI8sXjzfXaErhCxHOr4dtGt494tpm
ONfms1fg05x5A8fIrl4pGKMHnK+rFdIaNEMhqR3D3XPWozIZY7rMbEX8FBZpCDyN
laD5f0ahxjCotGoSO2OJEuPhrmbYb8txtiRlrP0O6vE2XFBA2pUNBru4uOnow+rK
yfFGmhuZ8WA3b9S7Flrz6apTcSBuqWsKo4VcM3yE3PiAEq0CVJrQ0xz+uO4AnY8R
tL/4MoCmRN0j43iFUCvTPgvDFPtgr5Qz448lyaj/K2Iwhr/MCrw8LuFYP1xWnuG1
YY7zypEafVrSdMKYCmYURb5SmA0L+2+aZwE06a400mc9Vdzw1puXhZSPRTrNpdaq
6w9km8YCfVUJJST65PkHL+HtuzpH2jhJIR9spKCL7mvCBzI/EPV4v3Wmpyxj8VJj
7IY8y6HJ4bEzL0MRsZ7toTnjG89n9Gjzesae+k1cw62E1atj2+LsWNTH/RU9g45O
IT7PPyAwtgztak+Tbbw9CXhVXdqgQ4H/diXKXEWrgOvQ6QI4bi2TJ6MsG49pNq4Z
HRt9vRrMDXNHOmZb3GRvjUWChzg9AYHTO4tRVuDojZ7ppjLLuFaq5w44bU+g1uK8
ZKqco8VupTrL4H1oPZ9tfB4NgofVi1M5GhUHyaGqBrmp5VL4F1/a2WWcheBqqg4z
RP6GPcT90lR0shVf3x0kJdkeb7BrhCiRVyk7+vDRrTIu6PvBL812J4NkPoMVMDUz
ePzGnlkeC6tWDB2H0K0bKEA3qQzcD40Qnatn3O12GRHAK9DwrfSOlyhkgQ93s4Ts
vhV3UFm+Np9n1X1xXpMIP91WIaIn/lkE9fEBzm99Cdl3/5HFirASfwgkkzGUyw/Q
6IwOkSozSpirsGz8iQCey7hFghmyTVrnfDoVe9YpWh74cccaAzzPMrUbjGIOfWFu
ublgyDtj3p7j0BBfz/xxp+2mWUNRsK8E8QZApj4xAbK779st8y44YJTuRzLS4I0w
X58zHOyS68PJAMHmshRTcsKs+Dgnm9OQ1QO6DPmRbE6UtHqkLU/zckoxC3vKh9ke
6XlvTSgSU9e9vzbLOOqea5mlnMCTe6tn6KFBC63HXzZo/amPYk/Q+y1yBHX9akW+
WrpHg41U8W84d/f4zmi80do2KNSUw0izBOLuaKj7EmBQEiN04XPxcEh97r/zBEAK
daiZHkMNtukpBZz0UfSMS2zrb5QfdzMPz9Y08XuFvKLXMozig4a3bq0Q4ie+MMZU
fCh7IBEyL5OYuWyFQPXOngXTBmIl015yQvWRd62MTzkqVGQ5009+Fi3hlWQEKLYK
ZPT0HQNmio417OUh0t7p8E5LEoBZsYwFvcMP0z5r0UBShGWU9AFDzX7rjAHSIob5
pR2+HLHJSn+LP4d+bi6QAYWiROnSvGviarWqjcOZu/wjEBNsACqNusjvvh1MdTZO
Pynmu+Y0qzpMtnv4A/taH5vlseezh3AZznVBQMDBLCeQmh1AP3jESyAO2ooz6EPm
CwCrxnqifg3DJDP9B51SGZQazIiacAArtbkVGdnGIltyyMdWVqxVURNI39lh36yQ
uNld84rAeLFliXFvAodLxQ3bGKw/QCCfT5zBxUEMtNbMwGLG9KJKoJjaKCK0RhKp
2+D7XtFppaua2pfbsXDc++PInoAfvIN26J497xWmjZLoEHweAnRVf5q9bnA1aNv/
vf7m3/NLH4Rlm8A5bJEKE5IMv1O1rM26zNb38cGk3OAjYZ0Hm86Bkg/VFqFEi5Tp
tcTUrFWHJVXcZi+g/TtzOaYY1od0mL37RtmTVoxWRInLg27YQUXpXQ+PjE0lo84R
CaxaEKmfTmk0guRBXLl0KmPwId5h6ToXFeo73DJmUbJrvGfP7Sgh1CmH2flWuExM
GUm2+6IPgI1a7h63D95vNm1xMy9emaanCrHX2l9D/cyM8yWlC+tED+HpYxA6hgqE
jRprdc+6jairmw7rKrV9O66Td4ODs3sJ70uNBpYi2uKxmfdgDyY4B84+/mJdN3cB
dpcjikFD5BazMGGhuoEgSekOXUFP3UeQxgXObPmUWk3WgSeESyduDl0NGdOmG3vK
+hOu/NLgYash2qkwnVTMOByrQzIkdQAm6ddcQrm/i4wJ7cGWZyTI7v4N4gHmhCfq
QT4dFUXEjkU0316uOKMAKCq3sHpZUhU8tZww/npmUHLt40SlPc1FAFy3IuPikM8Z
Ccj6CUpgEYtfvVFWStfxa842cfwFs2nYi8MjYwav5l8ZQpwBphOejqEVolGxDFCF
78/6+Hm267TBuPgouj9wIJpSdTtLmsyzzUEuyBbjwZitbHfDh6MI1KUeQGRQrZo1
RdraCXtgZWWbppx+rRmhwVDf9rzdgrD46igHCCEX5rUeCqzfWuE5rHF9DAVYqK3Q
M3hG8cE2sLzdpkW5qiBBM5ZsidAdWjukSg/SwPh4BSb1d/MOjbU+rO8sxRiOeHMj
+8mFfVmAhKKGkZp3Tj9XotdI67nMfeMEAWbNJmvoj+0WOXxV2kHBCCZp3bijb9Tr
Do4caD4AaBaU+gFw30pBkC5ZBLolrrM0Z1crcTnQ8IqVazGzo28DteEOqq554n1h
TM2fokbdL2bAMcas9ueWgoPsT6xskCXRgpuUyEvyCY++lIWeGpw11QAdt2mzQOY2
crCW9mFB3qmUqZ5W+GSg0ct2I5nAvDVc/o4PJzEJsYW7j7kIwrfOxfPXEMw36vdh
bMP8Un7jCceDQwlyqThI8IFGiWBkb1l7BupfCDYn9hInis5kj0Dwoh68WGL1r721
2CSkaVYv/tkqNMs0STGHSbJOvWRtE+HF7UabN8MdhsDf823eybrUM3WzEqTCPm4x
rQZHCt5RTXQkiDpGnLpxBMojEFi8/QpyOH3aPKx/8fMMHYLEJmdGEVS/jyHAEi29
lrJr9tpQ3oQLK3v1rHzVeB+7Y6lK+ew/VGLHckFyEr/L7r4FQmV7UtEfa2dlXovC
C1uSkhStJVAYkEfFx4jo4XD0uyrS1CX8hOcx6mjDF1QaMYx0qGTfCLvEQD0Rvhc7
LkO85TtPFEtVDV/37JLIDD8+6pPCsx46T+/ekhpbuEvDHp+IMcYJ5HE+o1SFLydq
fJaVBkAZOANwU9PTon56fdFGqW0qzzazZuvr0AwO9xHWbBA8Tcb0T/QrflOAlNHA
GRhdZU1HPRKpoSII/sXzCkIJGUCuNYmmAZGTNop0oFNDpRt3taKZQGl/Dsb4AA+E
o7e59BlQeMxRObTmow0qvH7slVDPd4h5jNGvmXTlSwL36vxhUENJU8ri5VVokrxK
2x6KYxf5WmuC+dR0DMCqL6m0uh2xCJOM+A5G1g9znt1W9AUc3+y2D7TiEkhLem8S
X9RT78d5Gq7uRHZxeNy5v9A6vudyg/76BP1w99yH0gcZ2UaocsJjZmWSSo1rh6Zu
4x2peCX009Rz0McCBT1nPIWWYrIE8IkZAt7TQrQOTOomC7zoZFC95Rb0YZX5OkDN
2D0vROU7A0FRnk3kU3g5foKV473+DNDrzw6QaAHkY7SIXeIL3iBNBLdcokEQM7BC
JJrMasUtnHwsA2dOkR/eDRMxgQZ7t+93iTYP+TDpEBiEkZMRfh91TdxGgV6ZcVGk
gFZo+J8AKUmLrI0KBa3F4hUBdhCNs0LR0iwFRjs9ig9rU+V7KDYW+b3Tf/8qnExU
/M34Xuoz1BVkwP1IrLHc8D0Ycr3srwwcLc9jyEJPlwugH85PPP69RXKgwaNm5lBN
6bJDZNPU4woGvthd9IMc6OP57ckStiPZx3y4q3fEOxeT1XMHo8FF4nQKxgmpWRyQ
CUxknmi0F1ZbsGhS/biTIBw701sjc22O1i/4X1f3TR6x2k9kGUUkadjk9r4vDXM5
vdpTjbDEBXFts9Xw9hQT8dGKuBKCCbjBlxzUqAHfT2U5t0HcLsgfMLPTdyga8Eu/
UKuv94hPTjXby42/0PIkT8QYrDHzbUDBL0DECbQkmTOs1G6tAE2Ef8CK0BRF97ts
IGgX2lwnZ8MFlEdz1VjnrxK6d57BuRGuJp8FA4KQm3DSdfdO43HPx8kW7uEzHS3w
9Lcg5PhGSOAAy5TRDjryAnKY+TD0ytqRfJPfoKMAflK3qzZyl9iCOqB/p6+RSKX2
d4ZJqJUdkhMzPoWIDCqdFt+1jqNAs/R2ZLHqDmJWsJEY4gqLwSA2tGwLUU1UYnyh
HPl7aYifzFowTBppt/NiRYojpoBuPlczAhPv2riXYmKiNrwmB1/PvNaxoXCfqIJJ
Av1tap62V7DqTXbWbOuamQwTa7DpQ4HQKb0pEptw+sFWCRLwd3yp8l4A2HT0/4Hn
3KoILztsOzrLVxPzB+EXnU0T9tQRnla6YYCKcCe0wBNBvYjnU96jl8in+ZuU9a/3
FX1UhjMJSbpcw/0MtuqrK9i5umxEMEX4mRNhdctieXFoQ5jn+9oohiGi3yMxs5Oq
4pjRPZLXmoKp697xfYZOxd9L+VK0Thi5m4ov7ZFtYcXXmMq/+3chxY9g+slpgQaA
0IV5dGZYKkDQVl8gbVpMIcYazFkO4q7HN7auBKIEYtT6DZHIZfjLaPqbysj222cZ
4cUEAYKaN5k41WGHHMFiYbxTpWBiVOWdnQfq/dFEM0qCV8kUDqUOqevM2v9QRpal
H+Eo2TcytxjaUoaLbW+uz3CWeKbO+vyI3kEc6WPc9UxiaRArEQgcJsyJybGYrqRj
2KauqSxYxVDGe1/yw/N+mSyVlpOowTibKDDDnGhQKSLHDMy2WhPFH7j4dlipTZoD
kHjTdDKA5xMcn6MJy5195Txm8FJGBAoFSAfA7lBsFWzFtVFaKDjKmoErmbLUaagu
K1j3k3G+UgjrLljY8h6YdyZzN/+JwN07Btcx1Cg6kuRuwi6xZPvrVuXtRGq1epSB
YLoPz+s0UWKg6HVDZswfv/2Jf2VYVu9g/6Hcu50VAVI8SVddo+cO0O1rqnsDkMOa
pfpXOPBSv31vehwhassq3cAl0MHh9b9/LK5yqAGXEIlxYbvFThvRdAbJSje9f72P
PNDXomH9fK4hS4PZmLaKui5CwLYDkzBImyyQWF97VqXGyOTO1W6yXNZABmBMKEFd
GqlLtR6CPSLhBYi3SkYuCdKTQAxvKuXcJY7Bh5XCWLx99REBO+8fo0oMaRAhUa6L
p90Fh3NZt+XYAJymtIYDNe+BCUCtgeFKLDFv6w5DFMjSRZsByVS1GQhldYWdAnTM
J8eT/9JrWnX20hhPphCEbpiHEJN3lYvkper6WvOU3Hcri0Rd7mE2oxamxR3KuPQL
Zv9qMVDePlsZNZOPbkJxDc08V8P012afpJQRAEan1XuF0J8NDub+w+Rx4FiFmsnZ
7k9xfFZHD8Sr2AQlfg9pu3Qv5cV0SKlfIAm6GuTDtm1XHHvPpgv2/u+bYXZXNuPU
W/o6JpXyMH6HskzMTPC9vRUTI5N/7fTzSuQz/b0oyYaR9IVkrewUt/3UjnxskLZE
Ue25HSazWJH6sP558RimRwZAuYpyQ84FTwTXJL9YCnONrMgE4pUuHSLgslli9cCs
HnMdWV/WcZhCyirZNu0Xr82zjqQxE8Ao83WD9zqaCVWanc4SJCWwvSxpKfb/IJUC
/P9LQMrGJmU+F1h45e/Tc3mCTgvuo1f99BqH1ZzVwtg9Isf4zYTAlG/VaWkgt0Kb
TbFZ2bWt5FdkcSFHRfvWSTmnaZcQ0CkQOPSmhDFPvj5XU4nNjOHMTu0oqxoHXGuj
23xatt88tW+x0CZ9t2XFCkrRAwqWC2E/G/chCQFGJYX3Eh7ttwRwDE8RJFbv0Y36
vtOUcEXg0q4YxMMRpObAQh7gCVKWMFgRktgWV8eukt88idkwN0mMiRVw/RdgPCdJ
W+dz8hRY8OkTZXApQXixauyRCu9bmIl8x1oLm4mtm757m1PzMvTm87PoGzKHDK/f
aBM/+DKu0sX7wP0wHH61Sjy5ppvCKmk7yStqGR1qZRZVmDUBanB/rA9VeQyUJ0Ip
OglSeruQ+oqWmBL42cdqnuoCC6VYFTd9abX+lhnM4cX9jRWVQgIvZPtZxwdriImz
tr8/wUdjvYGvaxTZjd4pQ51YG6kwFDz92xQ7XePR7OGF7Jni1lAtiQK3BS17PIsU
CFw5+JbbNBnpxZegFl9sZ+tVauOR6y0+RGSyBSXVbzSLCdCuwLozpvZHh28Px7p5
TyF89B8+9cdrjmOJjgZ62iI00zwB3pW6PZD/Bzi5n5hME0bHnWXwyocxw2ds24Ei
YwWbtGVcd7oK7Zht6HxfP7osJ7iatCOGo1flUrlY6OVtunkfEzoK2O9TM+dLRwPH
C9onotNXcMG5tw6MWJASFoA2C9BIAt7+zxqDs3x3cTNfaAsy342nOMvJfr/2M3yR
LT2xDCeUu75JOClZcrUjVbaNc38D1w4qR3nvA2XxSJRalF+WLH6vvRv2WRYq+G48
QDQDDho1J104w6EbOj9Uo5PTZ5v+8lZScAX0o9A0l3d6F8KJNbArFPftPBurNZyj
KTaSe8a6VmLVjxnXoq1D2z6hEjUcbI8cGLMhtvyJBdso4VFYYGa+ZLAqD3zfr7Ck
8FEfZQ60kGyMSBvoNls2M4agI9MwpT5czJRLSlol5AbkGnQ6S4lulqRYlq4w0Kwk
Ny/a2B7Hzncu4+pKpbqsCHEF3b7/b5lGMP/4lrrZtjwXVAZRqGT8Ldm82i9JHh4T
K4A5zRooen8UiqZWpZfWQr0f64TZuOa0HZioRcnYLHSuhJFknCmk5Tj8QbpLj3Pw
PAFyhb5OZ3oPFDdK34JjTfz/n5FZ9EcFMhLJPQWSXJ5/8v750exjy0FveZ0XCBQo
P/5dJ2SqYm2fE7PZ5BZA3gTh+0OyDuID9DI9rgVkk1zEQ4QQ6Hc4+H24/pWlpARo
YSuDGrI8QkqWrlz43snFqg9d+P5M7RJnoCwdEKGBgqyyyjbo1VXHMX2GaSjE7+8L
cFT3T9lELVqJyBiy+8OTqW7lHXBo3+whwlsSpU3tpfDQsviu9Y8GLjD+RgDWJRy/
MvL+k4wy/wTNRIa5pv+59Rl4NKUfoVprBtAD5+MxJ080EX2l9L03riX+yFleV9Ey
CMlDGO1Z+KG6xpKe1C+rWpBD63U+WxXujvSo1MNLXNOgNwJlXF+KjHwpKQHLLP1s
Xuz7QzczNzVq+nefSoIM3buDF2Jr8QljFasKjWWdmfzp97KCOf/KfP55A4ixdTIO
/qRH5z4OMT2sUZIbvjaE0lhLKkeUJBDz1Mth+mwgXdTKfVsQoVDoziQOvSfOXQ/F
4Zloqd56oqartzvcqasa/kIfCIH6QqGzG9u64GkcjEbgtTrz/M4ABn6kM2/MaafK
ys/MBp1EI4Tr4Y468BPM2qyZ4QN3UmShNHkFbyCWsjEBl+4f4AeTyjTKFJ8289Yb
sKAW94pJS/jBB0w2+LlYfEDuhw7VO+ic/kWphFSX469M4ufR6Ot1f+HxE9Gt1D8L
uhG85DX8wYOkP9EoKrmoOwoP7OTDnAQCUuLXDzDRDp7d6pPsT9b3IT7zOlfDg28x
C6aw12yKhlQrPp7oKns0hJuxzqoqFWz2H15lyEAHmiOCrQiqlEKsw16+Th9inGNA
gaKLIUqo9RIs0RRV0yNMqTvfH3Eu9gddNTFa5x9upyoin+/V/vC9tm/8b5TgKyp/
VRbWlOPGXurGO7pYSSqNp1ca4Vqj/GUrFLj5quqySgVRyROCzGjd0IP9msMqUXmQ
9iv4RtxXi589xYzwBPBfeKV8gtm3BKYlC01Jv/f82MNzzTwxtvrDiW1XG/ms/oNR
ULvcmnh6JJgBMpahGNR94iD37zHcFB6v4jkXuoQcLkKt7Ahbdb8j68H1j+EAimNw
0EbPciceSJiZOVYzOiLrNp/BsYhRu4heE8Uz60ZI+O2n7ry7oMLnkiQLkaZwLSA2
TLUzpx4zlDyCzLS+M7/cZGQRBr9T2HZ5QAI2pgN2JVNLI1QDXAMsakf0xTJOcZK0
f8BXwns2SONyHkTPGWCWRAL3Ueh9oL6I6P5hFMIRN7FlxeINKPTVZwHxQ2gKOEp3
DLxx4jQIoB+rRkU+rgKPjk+yw9oSJx3aXtim5MUbTFOZLJRgOuvDBw8BeWrB+n+e
lDrZg+i8rnAWoe0UIpAebXR8anr+7BWSzCD37OCDZSVPYmQvDKeYgbVtLdjVkbi3
H8a5YUVIB5vpLBpmniJVkrNPdkCgQRTF5Wg4Ljpkae2fBzau4yjWUgsQuHC6O8wd
JFRyiF0S6L9ItvOtqJw79gbHnatQtobiE3aYYOHju/EZqRrsS9ZnYfc1x2tZ+0EA
gmO0V1NhqP9WW1hQ/iXfwHOCZ/+Qkv52yGwmu1uFoQsq/Sv+HxlbXq5Pn0d6SDdb
kXRNoicz9Pf4atAyUu9oN6ExU5xzBoLCsJobbLiwCPdAFkWha1/vbY7YyZArKavZ
jt7u9GtLgWTSqMVMyRNvwTnCBOpEiW/1QCCLGv2a9XLPshGowO660fehArvV+CXT
E5vAszOmw/IuvmRLT0UO53bBUjZPyoIlWdpsH/kMfvDKNBSN4NM4G03kroIR/LJH
yCf+DWvFM7t6ennIKRFCvOch5F0YQu7l6As5/eKR4wWJB7uVw79HK6nBk1g7ztdP
YPTYXhr0ESc2vKZr8bF8dG27dHgAzvQlNoTzKPkvxqsSF/D6atx9BbbmVhPNIFCX
1EmxFEdEmR0u0N3HG47ryDGuBtSw/4ErDWlAIhYSBafQqFeb6jlW9yD7sHzBNwN/
bJv6LrW1JL/56y623NS9jaCD+YucbOCasVMVDEevrqvWIjczctCxUSD1HhQcpGf1
k293YoJITIQWmqcDgiG2XZOLCf/yR5PFWOOjyuYXkcbeOUVWlXB9EXEpQGGbPN7d
NNvJiPTcwJYFjlrDzN1v23bPyzRDMDuJmQCYgk+t3lff0gLanueOBoh4kcyXULVV
ZlkwBDAjSLVtw3YlqCQrKhZMnUP0/Cuy3YbZwg9damDrcOsTePqUxqSp5mIU2USm
pjGjMCL6hjE5l6+FJBa+4dZeTWvPZOitancliYPcTeL5t0s+IQyQqddSIVCIH7pL
31LS0YDesUTT2NI+I6idz4V9wut+2en1A6ny7AEXgvRV6fi2jLJveti7yvOo61TM
umkmko1x+KB4d/+n1OjDSEiD1SXNyaHzgKnmkGbZHHsUyvS57yZAqRqCT55q5hUA
QFd/ubX8C9HZYNCSrCHnKZHxLEDEG1wrii4nGK9qF+cDRXn5jju5gPmPJwLZDJFC
tv1Oy8lKsHA0qJFYC2D2fuGlyle++6b11mZK8k4lJlW/HKyUy0Vx5ukKQAw6+yQB
4eX0nxniFER6EzvkRd61I//tBMWSipUgvovKFOYf1K2a7PHa2GGWF3XP+ZMF633q
TuXf62CfzIbseub1RbkmppTk8GgaZgPbqexsHCIdsSTgJgJHpAoEsBnbmlQPle/B
kf11Ni9HEUpP3Q81r//NuXI1gWNv0RaLBQboxDQEKKsgk1VjduIV0PC2cwnEBkcq
vDIHJPloCxrfE8dic/aDzA7m3o9JPXa13SKbMQufZxbnT94kJaVvX6jS1FRY/pbU
f9sxSJS4n+FvsbxybVFGKGQ1RRAbMwGS/q1QJE4Nk+1pgJGvxgeFKVZMQbrCAWfA
IIaVaQ7RS8HmZTwLISmZq0fPYpEX0bsctowwNGYYSJKrtsovuH5WkDe7sxnmWYrL
yorGesGSUipni1F7OflflCi/5y0Mu0JtyLNBUtmah67JB054jPtUzDS1qBM5vT/n
CU9oBrLUc9gu2eVYlToHuDrwZKVipfdnu7H8i76Hr/1+7TKp3hxNgULUbZvBPnTC
eHWDM5A4ltz5tduwQubrRuaALzU94VoyIWkhGtMkHnSTQeQKriQL6eEX44EL75mH
Siac8YpiE2mv68zYS3KU3+ZBBT7bhLAf12lWZRobtGvMoJirmPkiC8W4nwYkh+Ed
RNfy57eXn5IEOOIU4C0/R3ShVO3nyx1xOdwkV04kizppa31d3w9qUjsI7IgxApMK
jHdPDoqG6/csL2famaOnSMiVICrfEHKy2uOtrLq+/Qb0b48I6S0R1yL56x1bnIzJ
XoAL8DYn3JgwSmLJMFJXhwcoGstNULQUTGfEDRGCdm99CE1r3NUF1sXVuCBpdiKy
7IqVBOtFN5SP13gQJUn1lt+azLl4KBx6NyNQ+/JWQGVVi66C8atE6MdbVlRgSDor
d/nC7flVJjfvnw7KXnN3J+XWrzoOYMflLB88nC84tXpObs4yOT87CEEP3Tyg2eNa
ODJflZC1c9/mPB5+fwnayTtStbl0UVf4Ot7UDV+q78hEfaeKiDhdMIjXRfy1BZxm
qf09xtTT30sIF8iNh5wTPcmNBQJujKz9oRm4Y4HgMjvV4g6ro+Pq0OnTbMp+xu5O
+mdQ5WrBmihzRz8IuSm32UgN2HwG4InpTHbeppzVBD5SpPKJvFoeOINvILd+GWGv
spCYLvPJPGvxAtgBv49VS2wvH2U1cn1pr6rhzBI6dApvTeatDGH/wUFzaFWHw5uj
dzS9G5k82iHgafPwcPIInPtZF4xH5R/XqNwqwL241M/vS3fRDmWsiZHqfpqJAPVp
h77m+gPd8Zgvu9NZVoji3Wgk47yXU9mMC7xJ6I5vrB/g/MLHtRAOOpuHq1g2JXju
uk+WkH30+VQZrs4xzqt7+l1bfqAWPxoYj7KrLE7HgyZpu9GkQbFzX4fNHS8+zBOg
IoBVBBMtQPt+ADuyqtJxB03h2QTknDWKCVTFzgj/cM4+q0Aqv+HDYeBiGQYuBrLh
PONQihkzIRwmPCHL4TYNjqCF8kaeGFfqYyUM3WP7Geq/0Le5nXWkXp7PxYoTmzLz
ayPodKMXu45MPd9Eo9wMHdAC5YzNwOk6qlDcT0zmpaILgFP47X1sa85WCFtR3l4/
IC2xOBS1l62FNo/va+gfPrEKW6nqi7RA0rvrkA5Gz+HD+Wn436RlNF1Fq2NMXXSE
jUb9AU339uERoVpGo+eONrvFctp6LAR9orm5wcZkXBeNLBduzgYnYl67LpboCVKf
DfLRanAb/Q31/VK6Mfc5xIkvoL2EO+vc6vtvm9F0VdjUVBzZ1p/V6cfXGNQL+eWb
lCpqQh0+4n3t5fPVkX0FVcfIrq8Y7LIH4HJCeeRpRJXiwVjWbXkBy5/5yWNr5VyR
qiJUHmo+V1qDR5TaTc3GWr0bNUZY+zVnQJw9xKHOKkLcNfxW4M84txIhqSgZQwRL
XBVHGD/odFZx5GQGyjvoK0AWo/D+CJSGIsBbtpTQ4rIHeP8BXs+pg+eJbCdYTSPX
cT9uOnsKC+rHTQbeWGPCxpxbhGihQgjRXp2wvwOr5gNYVa9HXUHN9kF5Ob1C1+Ha
lC41bXLHgPIOrpnz8FHim5iRD9/6V4K/yUazwnrMzntlQv+VU2a3CIVU1Gqlky44
6fSCyU9vQ+8QQpusUjDHEGMs5wys+OQpFRa34NZpFmc0kTDEWW5gRS1oRaXJL2Ty
3TogTfd4LAQbqLRw2eE3Hg78CBSsckQWvQ21cXHJx3RYyeCYNmM0bqO6lT72J3Ah
EGmkco/nbmzl4JftEXM9QJYMAdcdpBPF9kwllh2zVmtjsv+YJoMjtvh0UTwsrWEF
lzdMfJpXriR31WCtsQPxSYVpbGCxqvkj4Skb8wonn7XIgYQOcc1gdwqnSznS5SxN
tMMI4q0SC5heEdJqVBx0aYwDDaiNNw7yGIShIa6EyHG1KwhQMntfdpaStlsnIhxF
f8B/x88XO9XcL46CTvN4knYuQ3SLzaiH7T46/09p0PAdlCE83aMyFQ4IrNqhKXV4
MKdx4K/5/ZTXFLO8BGlGA9pFjCde+w8oE3b8N45N6O+nEeVxvEKhZUpJmd350gax
Fv2p2QJHjRdBJjZOzXQjhgwyIuJqpR2pWe2uiSietFocrAhQp6D6goP1i64DrMzJ
8U11YLoYBqbvubc7oInZA8oWQxat9wPkysHQWBletnOOy8uCRIgPKaph+dV3X95K
2D6Xqra3+XRUcuAPTYV43XwqP3fKMClRRlXyOnJyqsySLvsZMySZF1AV8S77veOP
l4rLdKafx3NYJA/4glj3SkAv3SliTKwIQ0wHLn9MbtWLzhRyvHQAeWVX2n2yMaUh
X74DuP22lSsJ9TozxWIbe0zJuGkux7l83ZmVpdox0Ugdby9kqHfjNguPKbQZCuJA
NWRzBf+PV61wrdCfP5a+iefOwh9VqDdc0NCOX5r1Bcb3CFApgXnMBzgcNWI0vH9j
D2qGguc10ws/YOPWRXUtGG4sBXZEkZCW6hpBx5yKm8hR/cGLs0BvmEZc7b2Us+tW
emrnOA+Ch/7o13ho7rZimb+4FFal6MS9vAKKaumMCJOuM5VPITRWSJcZAhacjH1K
/Sna9+SvAgLBdwiz16EG72Zs5HqrT1A9BjbFLZ7khBnEqVuU7UmhX60EP6jg/l9h
0PDKhK+B77NhY9Me1S3U7kW5jJ9ZmmvEEQLJ9u4Ipz9h6knK58ooxRHuyIvVUr9n
Y7eU4oggpZGEqTWC+EHlm5CNutpOBCfQPhcErRyMCjPiLca+q2933/6/cX3/drGM
/y+W5Jl+7z+ieRzVlU6Za40j2Yc9bTW+05jPQXbjfNy+632qbrOAWrcx6oj0ExCJ
DVfVtU4L8diCqT3chLUn5dHlexyhB1/1Ol0nA4qUytfJvCMLbexusaZeRUACix6U
6v8iKr1l3QwFKzZmqtBKkLnkpfKb/7LqKN6pybwcK8OiNE8jR3ppRtRDguOT/+rI
f5ZBAjR3Af17WsYtEYIYAlfh5YoY2Uu6CxjbyhIyauWIWBXh8wUhbMOk4qLyugmP
mCWsqTYPbFPaVtFlF4SXWUEMCYiVhZd0iV3yHt2gOPYXaSGVX/IIdIhTbRcUfaiO
7mA8IX235MpC7zr/J+fKnEla0Q2pPYAIqMrANN8zXPdwEKk8WlOy0zJtbFAXCopx
K6gUZV7J1QXVsdvYFtNhDRYhbmjs8GOxp5Hlo2xf06Fm0HBFW7HfsI0oOi7xajWU
Z4gZ4mHzIgyZ8fI5evpV1Z9IeseHtUcfR3oRLbkZ4ZiggI3Cls4KYCRF0n+AAvmp
ytotZyI7ZWIvDbZAh+E4S81VEfWo7KdnzJlE8ti4h0hubP8GkzTJnLKezbLVGbp5
DxklSJuc4kj6aI4pZ/q/DXXTk06h2eQezjeao1M/yz659o/FggbQ+SUeavEJOSlB
/DOomwPeVR9oFP/UR1q+fs/NQXG1ag7qJMGXB3SK48Mi2KQgkRpUifBwOp98YcVh
wsrrESX+ulpqRoFs4+cZElDFLbIfFyVijycRT6GF0mXvIULFP/74dUx0/ybHvidT
8pAzFwENdD7UDQ51MqCHOdonZvIIbWVUxGXR6qZXHqzfRTcqzmkWpbNsnjqsXehk
uryufbkP5YUfwLBWm63YTHOEoGYFNVGUx86jD/r0436XAOh3Z71Atqn7pTOZMvxD
Xu9aMwa3tr/NXGkNikuyakvvKqGD1qyMefChEvhVBNR9aLrVHqvrV27IaDM/Z+Px
e5JPySM8nahVLbTkzTz9JhPK2xq+4ATbusbIKV0nTCicLK7XT9PAekloocKNraEQ
wCNiPz2ll+JXfESzcIYrZeR8+cldGfXbfAfhC77DaibPjMfan6onKdqw5bgQnIJ8
DGYv+6yeE9p/tqmyJjloBl+HGk0Lo6tdyRomxtVrFxjZXmTlsY/y+wxnP46cZRmJ
G9KeCm8FlqOcStzQx9IMEEviyo+o6TR+eVjoVyDfDV4UGAj1NitqarybHAE7Sme8
D/v1bHD8ulA5Zr6sCYZJlf00azssIpPa1UFtBo23rdrnfW1qluTaCxsLvdheYCrA
2BJGgxtNG5XJYFPxF2nh+1RZdAyXgIFsdyHDsx9xAW/4sTOXMidMvHdvzN0ZgqoC
QSouoxqhorZuXaCIwgUWP+DKvkpnllV5iyGYEadumqvJqYSTGgEMZ7l7ZOPYLqh0
fE493/UyDhpTdTu1J0/LQzNXx8lKPoMlwlEpIDh5KDPmcuk/0D21991wvB5v972a
BADSAZimiCOP+RNwVC1V5ZJXb47fRAeLr7jwyW/X7olR4wS4+YodOD+12zXDZ0sa
pYS3mNCoAqFWKdtMctKBgbgZvWxAQ8IXoTORXlOMlT8/BArzAUODreJCtSL+vBjR
I/nSeGLLS5YqS0U6WCFgCjBIVKxKV60EKPxyckgQv5FjZsZWokOIMOjN98WNKDsT
gsavjVxxhK5rXPRhBR394ekltI+uj+F8TOujddOBCrd0+FW+D8JvzhKEub9AxDkK
2r89fm7uiLAoQ8Z6nRoqxJYppcdyKUgGUZ51uDvzIy01J40FjMYRSL5K7A0zIuU3
YyStSZ66ZezYtn9G9QSEKCsDgzl/e2YRfmxrd6QRvFwgwVUgWyDcVyGNNH1JL30V
0NdCRvzz0RPqoNoWDgPXJApKrnYnW7pSGu0AJaB1LbPeNlprvu6nzrEw2880/pLZ
1ArncuisdNYrtqBZtUl0W6p22a+fXaN+7s/FNR1e3LD2sKlZ/qA6P4ebhPGCIWCc
KZDpp+bMCiT0pA69QzectkiAhOMW+ZhRrPK8hJ5qjkUAB64Ev6rJh/zpDGssxhJ/
bYm50s6dvBZrNLTaZoWTKnAH+DaA8aMLGYeUSfMBP588lGOFa8IiJbptydwm2V2s
eeqLsgfqmnKtrI73kuPTkVeDt8ey6ExFuYkvSYRCQX2bfS8NxA3WV/nDwQqp5FPs
vsWcYp/86hur8Zzu7O1bfMDB0vP7hC1v1qmB5CEWBO71PGl4sypsQpU1blHsojmw
HpQU2ykoooc5qMZOVpF8Jyi5PbICfVQiP3kZdhZIMbPhSgnN/K4LNdPQ3JNGe5j2
JoYkPwAKQUG5V56b7QEzzdEqL6kK2hzJvJfLIZBzh3DssM9H2hAoHd8lhkizb5RA
M2m50zfkijjyZfg4jdda667PdcLZvMbv1vWa/TqfaoC8Rb23Bh//KyGvAW3DScpu
58n2W7+q64y3Mhs6+tmxMVD95eX78T45iYJd5xqFesH99ZhbTXXDZvCrvgD2wGuG
Uz79uQBn8V4+MPaPiahFnF2lggDB6mMu9Coj7Yv2+1apV2TtlgInykizL+C5GHyz
OC3PKyou6x55ZluwyCznX3lVD6i5AgLCEi3ZTwlR2FqJuo8SF4ZZ5MPFDsnKSrpB
6vWZMkos9P+m93OLjal97kHWk2brFZjEIaJRAX2EnmGbaB2OIUtc2ogzZpj3+xpb
qKIk5z1YdpgSjl183/Bu3fGDy5LHxCDbcENdz/lV9D6WvjyzwrOt77upihwQquBG
KAB26SfX1Uv4ahm4k16I9tf86JIL+aoCrLTKBXlOANtUiu6uc1wkeuj7Iosf0yI7
UsEKIYrpfVgX1SuJN3sXnuWnhZV3rlzVMQFM9Iuw/psupCsyte/dHy+HPsI5XKT9
59dn6T7LQRUyIQTTqmz/44UMG1W0TKYMKOb5tU1ZFHYGPVfmdsZd+Ea7yTWYRpsg
5kEE0sBVzA6+jdVillTIWZEGM0fG3YpNt/7P8W+RMqGCV+Sh675R9fdUsxCg4oDb
B5tVtIhPsHmP3/+7o39TCdGJ1/m9Mg8WHNiv5Q2/e7sWGubckYgUip4rr87MKvaa
wwixwPPWtVkgN5qi90OYkT/kp/kEUaV+l/KvkiFy0Fxp3LMe8xw5iQJjmjRbBtxx
A4kcNrN06mBNMkrfaDoPJgkqNQBUi6a3RUe0eFmaVWW+ZofDsg6dbCF4lE+bxGDS
bhq5LsDRCuLIauzY5Rxp41R9AH9XjFDrPwxyKAPDqATkibKIoKoA+NOi70WlCKnc
Xxm73YaOtRRYL3qqBRZJv+XF494ajvIVqwLZv3ONIqJwGWoKxX7RG/lOUBXna5XL
XqKpXNfgEZL4YcO/Cb6oYmUBeWz23+HJj4PwwU1ShbhlEr6Sc/IBv0AZe35bJm25
dlO+Zuyobu/go67rtP5mBVGYyaScFW8+j+cQGSAhvAvG4QHqXIS0x5O99Y++1hOs
hMOQhg+TD8BJcQrnHaMnzkEClO57ZyuIZ5+YPWDYAhdcIcep4EjOhYT+gIFD5A4/
F+1XK5p3IdquRyGmjI/dPm44u9ToHN3vS6JjdSATWl+B6ARUjtrFwTUP7fcbyzRx
LHWCkriq3xiUzNeTX5fNVNS+O1tUks4SNCGXxWdjIbcVk3nyV82FLFEJ2VH9vjLn
Xw7wb5wrtYmAFU3FIeT6JZkW/NCeOEJevlme8Or1aEf8ej4Otqis4cmzWXy+r4le
CO5/xQpIv8TvuJ8Oa+0zRJ9dKP26LgDMJZrxXxi/beFyYhnS3DvOsOza0TsAQLT8
PXkF7mLydYK+QT6OpBmaanXxDJ3zDcCeeCqPLGgoMUsOZdVC8+KhOSZmnoPqfXhT
FGU0ziOzwgOga9wZMiyQIeggMbmpm7ZTsK8fUzztrb7tBuuXr5GA1qgFrnaErhGB
ZCEVVdIRM0CSSt+YWw+Xega1osvXombPtAzRMY3HV9Dc6sxij54oAyhaAvz1B+52
48jqran5FKlp5R+hCGlT1Tlfxt9RxsaDmOMI0iu4y3b/1EU7ajO5TBCfnzJoizNO
T/tKSNJiGPic72A+cNAR+P16uFdmKimk0wDS8VcPYVspT7RzHvKitn5TuO7xvMAK
++sIP/iNyyuEySuSsbYLObuVB7Yl4JXQMUhHp99daqm60xUeQm6ENHsvwF9tTden
BlM49oPo9+fpeCW0FndEe6+tiPijGiYV5MkNWabmF7thkkpYbatNZtShjFqT1M2S
Jy7YkJ0mrL3Et2lrqJup4YH70iiu5wAZwfaRv25zWgkkML7ya0gBxScXsQjGRPb3
+BB7rLLXGqgrWC0K/swl6zCBqvdWBIsdIPzd0ABKvWB9EDVY8nLl9lI7k8fPSKgh
qVUtWvpPAkJ9WMX7ajS0HcgSZHbrRSAgIykKOIjIK9QNUfizo2zvpkYvtwefrHcr
rLCHgSYVOJoqpgmyEqTf7N+eOiD5GIxlHsADlWNTC2v1s2qikjBf2k/HVKyL4JjB
YPsVlgaGPkRvmoVa8dM2dMicsJWxtZ8OYPEKAxhIK476yzgxdhDg8bIvttMwBtG4
qR3uCD1fGfvp9fkS/IvMynw2iAXDpay/qyRvk2PhdVUxEPUW6F3SBLSUAd9mCk42
UpM+fttwhNY1qDxdn/pck5oz6qxWbp0rh984rp6HbzjF6Mhs9wW7V8SW0Qg5u7Ey
Qgc01N53MEw1rzm32cCc8mftEjNW7FemnH+h0rLgo6H+0jxfKuW5m45bMASII1SL
Q0G1ABNlhiZNZwegs+BX5Es5ZomuFedqpfB78u88c5sbu/KKDw/s3wOmdrpe+ojM
JfnResDuiXtAw57EXWbnDHeeikqZ/FqWu3H+R97rMuf54TbMK4UDRxMc6LPOL9pQ
LpAwlKUB5b3ZBlEonhuSiJzV+NNoOVuNdszUSWFixHgwnyVYeRhKrwyOnMBxaXak
tzrzncM4AIZytVS+nEd3i5B0LX4urHvJOMXAnPaWKQMNHPOrImpu9Zmmf9y29kYv
0CORg2PGU4gB10ExOarW+ACBMdVTLCnfanIhgpxi0swqgfBc52EKffkl1ReMy5rm
WCq0Tv2s1u/BPjrOvJ2fD668dB0P+t1FK9W0vjO6dkxXDLER/5/ecIhOvT/xkrRO
MPkldcFjorv3va/O4sYHHee0NVIKR0PK3IQHdeedLlI0O0ku7yLM/Hyj1DejFnpb
M6+lxIhb+7Jcp2j87r+AVh8NMjjg3G9N1BkDzV1rVWSxc4/rDjWbifbTT5ofgKEh
HNSi3nFPpPgBnYJgAeY11bgcVML+5uPsxV9gJgZcJGuSALCigWX8FFi6h30fyrlq
V7wCd2qUpaAv3VNiXua7/ILB1yOhvsVmMz2mdA/WdjKMmUCMJrqzGJaHlaphwWBd
t+tlwA==
=vJcN
-----END PGP MESSAGE-----

-----BEGIN PGP MESSAGE-----
Version: GnuPG v1.4.10 (Darwin)

hQIMA77xv53FwjH7AQ//dZs27Mkwb1DI80SC0Cc1DJXypzJ9qdliEJ2ka+z9tn0V
8FV6lJt4i0XDwrtvNk6ZMTq6u80fQijQzjJ/T1/6BuVyMcmcLNQ+8CDNcQjD+iVw
9N++yWvrBPHygPxxKO2sOk3wr65mNKk7Qchg2yjAEXukVlh6AKBge765SJ3SdB5t
wBrGHAhdrd1dRkSwXykT2J/jJZbMR9lVwMD1t1kd4PaTR7wVnj0s0g641WQO/DNK
TfualXkW+bAx3ZJZySBwAZWSmuLxQOWVN0p7RWuVEKQ3tb6JummvvFkP2cWTp6sa
zBxICa8npxvdMPYvewjBTBIiEZzh4opL8gr1cm2H8MC9mtNdPymy8b+oetF2XqKh
mlF8GnOtfQ/Dzxh4aZPr1b/pM6S74rGvZzNyOjWgx2hPrkxNOMby36JwPGfVS1bu
GSEjE7pS6JtmHpGp96U+tHM1zmNntD0bnk3zEHZ5ev9IXDQsj3DVH6DZ7574km4R
k2iRS3627CkZjPG+unKdhBu0ykcNArORmRbc0kZTWE3AGMb6JStIQ9C4O6C2LNtE
90HCpZ44xFgzKqCE/fLEmDLDtEzs4I0hc+Sr/CMbpEzJfhhfeOp6uTGDzXq2MtFm
c0jkwuFQgQYa97VIQtLISNMAStzB93OEJ0pMNh8PgbQkNz4ux+WvPbsMhjYaFpHS
7AHZ3YkxOjzowoceZh4u+X6y7u3gjRTqqT+2tAm1X6RUl1gQXORDGDbzSIv4p3TB
g8VwcI1D6903pa7/KLM91D1AOeV7SAYviLMTGl22YNx9Stb6ezlFg1kgfZIo61Ag
pFuqrLamfs5+R2WkZEWH8DTQ66WX88LSituGHFWzdDyXwKtyCLn/h0FiusQBWjwK
d2B/6rH8ct9Y33UoKB9dYCemGZIWGocySKHAtShLBGm4kngwRY7ZVq6FJf/7Om8q
A7I49TdB4YTPsHo+cRNbFN/iF4CH/gutEFi9KtoAFvbStemHuFGuFk0Y4Lf7WUms
Bq/4AQH9AbmBO0LzDtm6OjEvPKyfok2wzjvOYf4JWvpen6CE4BaPxrYFJvNMbMVP
znUoNevr9tp/eslqzLinFKldrKoNpMR8lmRstjC3RlR8tFGToGPF2uIk73yh/Wqu
RBV2WPSVdDyIK1gGcG6ODJBNckVV3ehJPcG+rHYMntF2mC/H0Gszl3cPZgdYPEjk
SYkK6AFunj7J/DTDsO+oVz/icmRdZ/8CwEFwL9V5H3YkSU3EKSNLHhsKUqWPAbAH
Jj1w6eRKVYdw8VTS5MHPb9MJehoqX5Wj0zMcHdvbqWXmShnSDhm7mk1JTSwgmKur
DoMyt1Ugs2OZYB5eUSwUhSrYpQdQ6iGUFZjh/VtuZ9Libvt4vDPR8mIYZ6mQq82Q
dAhRmKqvCSsG5+B7B+XK0b+yB6JiHw836VytDd81PpNx6ZJj6Eo/SL/Z/oola5ZS
gLgW1bd5HXkJgqgkuotTLLvgB18YJlkOpBrhDU7p0eSnwyXyIvf43/bKkywQNuKk
4GwRLdw5FHqyoHuw1OaUsRfIvIKB5d6uN6eUekkTxBcYOVGNwgP9oYkBPV1btGqH
0HtQ15vNxXQjT5FL01vkl8euXvoNBK5lkbW4Fw3JHUmhUi1W0LpayTZFH7Uo2gGe
eVEuQhb67vjv/gM1IE8H7zXL8QDJeCRFA1j2YcC0z2QWheEkfqb57N/GNGku1wLu
id7thejLwJAwqKfcgtRDBjheGelNCOafUfE1zi9PUqfW+JucZ1tUCSEOdXacu9+6
9cnzaRuYRapWneV3qRzBYJcSe4IDv+NsQobilJQE1DeFPOMIABwsXhF6p9iE/vvO
rf4y307wcH0MnNrf49RpPyw3tCgwnfbpNgoZySFn7tyc7SnUyoLdJNEPCxkYyKUz
csKNwgAp+zxWAsIrtxNqu6sjYaq931U3g8G3pQJshKLjSPuCHVDAjN354KnnXupK
CGJAtX0zuWaTZxj6up/RyfyHpz8lN2aKwCcUpiu7xi2ZMLw5Jg66CTcdI4SMKBwZ
5AZ6C+8gZbeDwTV0eku+rwoMYike6tlMkg7QhrqVx5i8LMjFXBvky0pvdlQYWwWB
Nb5kbYXtkun5LkgNuANY+FcM0g7jZrauluT2FOjB9lOy7E2jlhUUG16Laat9jBDx
1stHZcfO3vvrdraz3olnmlgZP6RFyzVu6gmAlXLqCRy9faKqWxkCSoqXG+X3w8V1
oNZY7eG/qkINnRO1WH5twbPu1xFLA4GLmUNizwL8s2nMUlS191vA5QNYEmTKSzHU
xEx/TJEB2dZTnkcyIeEnvZRCWq2zzYoPnst6Og0NbwjkEb04g8GuldOEj4G2j3Ol
f3WfgVl8HLcN1iixUWM7entTI2uQTvPILowvzvUSGlkmNKCB/ZubGBdvxS4NpnD9
KcIPv2Z7MQ0a60H0mzbYGtiV/9atuxp2DjRUamNbTl/wx4Rp9jciIK2qkX7+TULg
3LsSK8RhYmKWEVs0t46/mG9miohtZ6N0fyydJpYL+8Q9j9wn6GSNUsL+ftCdTa4v
WnUTnytZnYMahoUTKoT7iucdasGF/93Au2JugsVeGpGJjqjVuonMUKnDtjb7rqqZ
I27sfw01MC76/z27kwomxRTRHq67QaLWFJ9Bkjs9ErNhEfaDhk8HlK84J1QvA55W
Awfbe3BHm2T4mJyX9/oEbfHa34W2v5XRC5V0ka3QhS3gk2meey4Pu/sYIZheaidU
RWmNrW7k3MR/S9z16KEJbL1D3OK+qBGYIgKrrlYeQaN49SaMLCULlHKyulWhYmGQ
Ap53808V3oQF0nXbb0ROCw/UPBLa7M/tbNCM07dmkMwTtdhNAkm0I4oLQ/jLBFUn
k/e4MPZ+CS8KjkQ9+TQvla4pPSCiPJaK08MW1kp44criGnVsf5K48islz77dv3X5
4L6GTfdLgaWaANeG8rUmOUEK/h/WZoZifZ0PYns5d9uMyDn5mfpgEv0+tvn1VCnp
GDgEuZK+2jYJTDUuCAiJKKV26QVD58CvPJnbED/Vz44YZl0oqLml7+psCruvugte
daHBSVtfaa9XoikVwqdOZnf2bkW36vAcT7ZVjB5x5HD+SVolpHLtkZfRiWfsZgB+
0rFc3hUC6K5p9/7x+B7FDdW7I5TfPx+KqCivSYBaiWVAP8aNuSPCj3nvbeUcpTiS
57grRiYgLuqShND7YodNO7na1clWOQU1ugdy0Jdgq7ZNmAyoJY/bjqLcLZIbT3CR
Lt/K5ATdePC9QjMs+4TM2FpR08SFXxqduBKVmjvinyIA/w3fv8HVx9Is9JOUHJD9
soyFfM/CvvsDKoZyQX/GxdH8C49rtgVpIFRIuZo7MtPe2i+fpZ/zCWICUW/L9Wdk
ZHeZMNSvu9JtbcF0E0n2dScSLy1LQppk22MhPDIswBIgo0fdL6idLEXPMJKE1tup
Yv+NI910pT3adapr6USCcHVlBx8eerMYvsvcYJD3ZKSu8Fbhd2ZFq7fJtyduVSqs
aC7/3czYb9ufp2iP7MhnihQ1XtQTtp5/okEpBM0Wc8tzC4Tn+e7H/RTRXH6HAgab
u47XLR1YGIfVJzlfcDBt7jOJRrF5qHIN6j8fCRMm0Uk/qLzMogeyEyd/xqpmk8xQ
b6TlV03r1WpWcxOhZSsTW1G1K7ruqsknTtnmzndgRiZziDCJ46PaYbWfaSTXVhL8
i/1nGTRmlr9VyMpQHULp7S4tPJB1GS9hL7lJEoFtW4HMAMnLCWCWtZv97b7nTWL0
TsmuLqnI3XQrmCRl4xVZkkFhvFKa5ITAn6pA5yOI5WZQgDRYK/flTFYZcdjDOkc5
A/CuqGUU1hqQwEn+0rSJAItSKLLHzPDbxWJrJlubP8+NLKNV8TUFIN4phFxZdMni
1mBMa8P61RP5W2Ll2Oi+3YFdxC2ZfEyLRUxjJELuLsA04ICIbWxahnePOR0eSTw9
DIoPw7eW6PcV8t5ypYMzwzllFliZpo6GIl275IkP3Akr2YqApSV/Yr5jij3VvB8/
zTScCH00UCQlfiZ6krwknyGLWa91nV1Y76kBky7VkKSmrGQhFq9mHWajxTcfC0K5
mfu9F3F+cwhFFoQVuOMH3TVZ0MGm8dXJvyPGDBPHDZw1QniDDM1vQiAP4Jb63eOg
bK47vumwL08WY7lx3xJbevR6w5SLyaB0sSeKfoJsEgUeRrc5vyVrDuRwC9yffSjq
cerjC+Tm2CqrFHHaCd82uaeNQJqMMpMNKzq3Ff3/jY06IcA+xccbKfRjFCXYZTdD
lZmOkHBMt/JLNXv1z1VeskolUr4jT5fj4ISgzPJfwMm9vYiuNq59aWw2REX1qtue
fGJciej40/j+fv213YqNXNkRN9A25N6+BwpZqH+q66iKQk86T2hFHz3YHNiy+Fx6
2cTeNMHBFdzy43XdP0koR6chb5l7GPcxmyjKHPFbR17LKI+W3OJZ2XSWm+h30kZe
xhvM5cyIxe1Nz7ZS/p4MBfF5KeyH6aIiGuROl9yMo2qcw1mX27Pny91k3kNMw1MV
5LwxHrOZjA32nsgWID6S2JedgRVs4FPDEyzt9z/vV4L97YuA3WXmhcLRRi50F3tD
OyuaJYH4zjKuk6FXwS/sYmPgQxfgp9xIZoXBvD8CWoxxNZqwG6yKeaDobb0mmnUT
1C2yTl8uFkL6vUfMAEBOoJcfaeO0hZGyShWsTmDQGabNCFnfAleAPKuHdsRX6Dcl
rl6rmjeJHqgibGJYG1rguEx/tZaJ3/Gh+C8MsZkFUeHGqKynbHVTAdahVPJf+slg
xomRzbfFHPDTuClNm0HFTTg60mTBNwzlo91vZCShEY3Kgiq8V9QVlmlSuko7gjhs
i4asaz9fKSFxOKUkfozISPvseiB995JpP+JmslHqIJbIJVhqFb4hKUYGJopTLA/O
dl8O6YW1ArHHUH6uhMTq91aU9guwaLCJDl8U7w9M8dP6JQ6eSWI4UqsuB5XxKEVl
jaJf2h5mUTDW8eEfinW8NYM+5/MNap2/02ABOIwsViancAsMgqi2zkG1HPPpJoCY
BJvdECrLzWwKXzWyehJsdSuCTfuvOmbLKDkMjPb7BzEWSwqE51rJfFtb5MC++wIE
AlRLhdXA/ZKPRsytqATSi6NhBjFRyCkz5gMDpO50eTvoL3Jn/QoPVhJK7JoZQkrx
G1TePhWkDNjhsizVKwIDT/dN/4PWRzsm2red4q3sncezzmFuywuh3QipkURNSxeE
/LRSvH+YL3nuzZfAJ7W+KNQMw3e2B7ZIjEs4CFtvOvVOCc5QLBZQBJB/odfZqUOz
hAqRN/MDgCPmb5swOS/FpbSfZiBeUdh7YIoNFLodLhKUJ37cTw1XaVP6zQaKJhVz
K3E4TqMPhamX3VpqF0uoS+5q1o8RoOWfCJYOHIX+5l5Rzu1Ok39Q/ImfzlkdR9Tn
IXhSRRj2yFxHHKlXrbToZvy1dmNL2c/bThOPfLUViFyrWCYcCSC4qu2mtKKtTLuK
67g1wlloOB1gRg5aOKO31Jo1GgAmSIx2bVqN+5wZ8kJODSFLzq8uQzgSQDNBMBeh
5Cw+fRash1Uta0DYDfza2QY4tCu3gqrIgjfRJ0u12NWlHfzWD9oVojLIYhZjpNhC
qHlG+sphH9HuCf6R2fclbOwmthxA8yBGQfbbZ/Y6xnFvVpGz4JH5ErO+7nNVHkSP
oOD7saFJ0mbA4mMF+i2LC/BfnHggHaIKBhdn2kcHOYOr1j2RNUIGu8uX3a1HHVy5
mqRgxhTgkqdy5uJsTPSzf6NwMhVfCQtnrjm/UfW1i6FpFqZ4EHJSe1YImTCLIax6
odeO1o5YYQTz9XFk2qSX4PbMMNMw+SWsJBX5mvfxKNE2n1ycIEhAGM75KV1Q+485
CzqyTGTqD5cc3MAnFBYOLMsDAYlDIKVNPUvIs9NhQXpZ9RfiT65LW3YtFi59nAl1
N8pzRNgNB/8CIp06HcVGcHHc+gZSFkaJnXSIaDPSApD4xghROf6ElupToMdtOrvc
Lw0ysKpsNfCHlNZD6JZatQBKnfSoHCDP8hAsO8Cu8iBpcSFipiq9RuyOxtyQPJyH
xMYNczJURphn3eFwQ2cETPWNzhj6+zKA2NIAmaMl6H6JwjkmOvsrZ96UdjrUv/f+
PlIcqtQ6++iHidui/eFX67nr5897eKgXEiPyHyAFjJH3Dfpt4DSikrIMc8a7mQnn
QL252Rc+RYCBVp2oVQ5GtEEwPTxY8/ngmOi58nCw7Iv25+1mraGR4nPROum6tEIX
trpjvBSCxhLgF8DhF+g2usMMqV70slv20T6CyWFGNMt+MfxxJpVYgY+cs8rkCApH
S1efxtqJT9h9wtwyYCpRkGesNU6WOuTOlDOYhXOF4WXvUDzTBLzAz7LHINR8XuKg
SgYY9qw+lShkWXpVXkXexF/DE/2ZBuG8iNlDq8V1FkPvQo6vUwsoeg7w7lsOyOf0
rkN1n0OikidJSSNczIjwEIDmqzn0xvXuCZQkjZOEByqJphGUs0uWLvNN2b50Q8rQ
KtzYfiIq6nOU1nBi3K5mTQfm5p3DYFzp4/poxkn9UcBwj+m5snx7jEokHjb7bvpk
cJZLPl1nBXxMXdsd2OzOT2fnK8MmUPqrJETNfSv7k6Ix2ZO05seppa0ZAOqkhk6v
OQlQbsfHyuRvG/1gc1KfbPLSEt1IjHpl1nShUtC/TfWPCVPY9d/4T3Jjke5NNcxr
kCREOrA8Mvkj4qaJXM7nlEgdoCwY4CF1w0q2ACNoUuh/ebp4AYnb7z5neRf8ZJEi
gwQAO7xMG/kRXTrvbUnJmNeZ3qc7vkeQozl9BooY3kvxLZ+V3+inHsRTvK7jZG+e
xNRMcp4FJUOsR93GtTLub9fqWlgx4l9TgeTkkab8GVl6CnOz8DBpAlukuPJ2m/Ft
Bdc4OkqJZYNZGuCgqDbd81i4u9A4wYZg1Aw2f8o+3OA3VRyIwjkCjPDxBxorI9VZ
M0J2W/srvVcMb7d6jNd5F4gMaATVUUKbaff0iXtenXVNmF6v9eXC+U60IhEXToXb
M8ySJ2UNDvarc6qUqXt6aOaq5PXJ93gScNP+glPL0biqZSo/v8FvekD1BFkNiQAi
2aFAiNEMB6q+qFdgXjCVgta/ZVGUL3gv3KmGV6slJuJKPraYstM04D1E028z5lMy
gpxhOpmgiVOKJ+WyclrNKEVAk5D+Dojv1qtKstdp25rt8yDn8kY9YD/VllyxTgNk
ZeEkKraPMTezXck6kZyC/376ZFsXZIAs5UchEDrlhtxp5gqjBqMrD6w3u2eiYYlq
VdnT8/g83hlwzsWM2P2fc683OmBMnUB0LiNbDidEFk7o/HUYMi2Nzrrje7I+IA1I
fe1EV/qOr4ooB7l1WwwcmcSKc0XlICkcN0QxD4bx1HFpzqRyj9hd9sggvEFHHhzP
uzCVF+rwtumvrBv1KsG+QETA2gYhTMQ2ARWn+BinBlrZB3j/CS4ApZDoM1T856Z+
ADynUSJdF0//dHJHcZaXZ0nTGHoU1TOu6UzysTEr/pkdNv8iQe3qA4qO/vwpg6WZ
Imxug7pGa63k02gnrDh3vOcUNTqv/9MeM4wBYW+hJGORQN1Ni0VF2MxF60p3G02j
VvjVG/wTC9eVo5S8FwumQcPWdT2uD+Kj15U3k5Wa8eU5M+mRCy5WBPtMdtQ2XOfD
ZfJRQvLUZXRVRcM5rppaci+CtsgKzCHyrr41azRfpeeRITFVy/B1VkXc//oPtdRY
1QTmOvB13axomJn5Ec4Cazd5kGoMdM5tWB8rk3MFzKrRMGqYqFr726YOzdAvJy1T
q1klVziZU9gZjhCn0X3+VYrJ6GyNbHSTHnbv1EP0QRVAGzZgeDHE46z/mQVP2WzY
+zGOOdNyfRUitDVZ08r7esxJ/Hmubegwat4FnVVbakBUdhYpzKJ8wo5kkAkU2ENO
6ny+5JQMZkXtP2c0LiJo9g6zVGP3RSO39ARlujX7adcM9mcFwRIrVNjSpA9dGEpO
gl0h6GxRbZGkSwObwVp6Ak8zAKs8VjxQkTWAB6bTGE6FtjyuqVHaZNuWMUBt9jgS
FdpN1yIfpKRxHzR5zmNOTTUmjYaTMj0NsthL3GWuUMK8hDc53kHFLIfVkt9gMvc5
BIxBgWh9jkR/oOKGjOhEtf/8Jpc5BVOgMFwLwlbCrxeVnf0a7Hhxs2AivhDB/dHm
619LtSeucXIfiRBiAe4/CZ5XgS0eMVPaIV37arPFkEEMS76wXOs7oL49j2F3RtmZ
JM6vhLDMFJblwd9MlKOTPVOQSapM4wI54f6cm+LhaJza0iBxsVr+Di3l3OLsM8vB
KeK77mb8I0rLUiYGKIFn7Mn6bX8IrpN6eZkLCzsA6SWvdNQInIkh+BT8WaKcR1IK
ol7A7h6O0Ki7KlnOnGM7YTEkWU82RRu1U46ggW+ysHb0f33StKcrX2Xs7kwJyHgo
/n2VgPkp/DSuqSqHuVpR2TH9F54BUqUSHdRLJeysDEVCtCUolhgs65bUDnfpIOnV
v8M3TqZYHpOiAcNnK3OvWqopCQUmwIQrB4d0Vgio5P+WpUmh+R/Y2fGSqgfIg6FN
A1C0hDHwN5U2nZJvMj8fPbODKl3naTaZ8w6kIjMoT128+6P5LsloyWFPYmgJaif2
Ed+u8KYPrbPoD8EPPeYq6XBOGN/wuhDWUIYfcmBgKOm/250EszhPEGidXh2xac7B
4xHlV1VMBMmokxc3dtyd/jYddNNbYiS4bXjaHnwzZOiF7MsVuOLJFLBkcmtf043Z
uAU6gInn91305PXoUKrnhbJJt/cbEkpdXPfRDDkffUdNG/f5oN1HF3ELuUfGllNd
19wo7QkpI1HU+odpQaMPI5o9ZXsOiY5dpKYFc8FQ7A96YoggWpZzsQdrGsV2/FpZ
kw/qi7dujaohQstQITNVRgAqmf2VMNi84lQANqGdMxDbXqtKd/UEytFpBSZnFNj9
mtzlcRLGzMr0CVSdXlwGf1f+bGUmlAaYJh4P2Ls6heiWcHv7gxkQCaUI/WUPs5ml
c/OsnB67ioTmux8QDzflH910XwEy4zMSzzOM2ziLrk/Pcu/VsN5ckK+/i7+8KHgS
bJFdAbpBWS4UBuETbVL8VG0MJEI8yU2aI0pvnX5f1oIEwodmH5bfA6pAxAJjDaEM
C1CaBXWsQcEBf6x8a+DEFodLclf0ewvSbTW+W3z5TZjsWkuLjkaqtDT8cDNfPvWT
U/llnFesB/a4IvXVK3loxz0abm+7tZaiT2HeLjydh21DJXF7/0NK3tRcuVsnTwuT
XI92nNtIyFUqOTUjcvm70M3L/E104SLECETucy0ELUFEMb6kBOmivAHFNjhEjq02
wkMz6Q/VGuv7nmWt03c71yYHB9AFP6RNOxJeIuJ5N5PD+AXsMunvZhJ47AZbEnAu
EgOaop+8s3dktG9i4IWdafi9mTuGtaaUgZUvr7FYvksQ014I18g9u5nh7W0yWdun
sUaWCbMjyPnNi/qlEpLaup6wdaVIcVwfJwNOYkJf5n75vGUG/SHgXSIDFDzNi7Az
s0Hkh0fYV8Qf/H6jq+xHJevhec6P7w1bZz4haMQImK5Tm+E9AlvLT1+Y1hdyl1Lm
QVpZII0stek3owOQkLZd3h8N0QvBe/wQ8P2sZaaBXMg82PZ6pBJLTGOKNKExqpPc
XVgwJCiqSgKpfKYB7oK4dbOPjlWyyCW/6GE3+A3KJW88LNIG8Q+FBVSrhUKUaUox
bK6K3w/CjnXESmeOUVmmelRUKWpK01ZvMjwIEhheXvPJZ3VfbzvrF2pYl//8C0Ap
fMx9kNnSwGcKRORl2bGHdE+sXAZ46eYmUBRNpRjcsdhOEec7Wiy2dFJuHMhUwUxl
ZcnTOONxBDrPyCEasopY9GnZ21nUsGAZCPbTBbFMYYi2H8y+LHblWgd9ZewaszrD
GAwiLmPcCCgEojPd4Z0rxE/+Z3VolPu3l5bPbnb26FvE9SoEBgba8xv5Krp2MVHs
gnh4Rhy4VASZggj2gU41zXP2m371TWAMvKI6+Wl6DJof0O+Opjnh/1E+n9DypMr0
cLRr+VM8+m5BvsDVAhavy61mA855XAtR9dh119m3dgnG1uBOFECbKZcqW11zqoDH
ljSMx0WQ6kdyQyaEhQj0WfXWMuxMvREg20tT0iQPBObwo2OQ2yeEYmxCkBpfXNQq
7Gww979m0FqEgseOH9xTpJGB/fMmZPD7MUjSuZWZDMKiCO1tSO9tJWvxb5OXNDu+
EdjtaQ5MPMFm1cFtdizKDV38N6amJaDYlwEnF0MkDk1yDURImOKb8LpVw0kZduZK
2tdb72WZTY66tDquzmpbtHRRdIUVCaKysk+8EBUD6lkZdpK6uOLQxMdXCi5HbTB5
Sjpn2YH3CpqfaDssRHT98OVkjcrf4SctVEU0cHKBu/CmnHT2xuERKSb7vcQ/yMxp
DhZxl6q5VBi9gcI9cxYt+IKXblqz6Qr00PZE1i43gtfinfs6orO+NBSS
=oQ33
-----END PGP MESSAGE-----

-----BEGIN PGP MESSAGE-----
Version: GnuPG v1.4.10 (Darwin)

hQIMA77xv53FwjH7AQ//R7KcYzY+7fAUiLDxHLG37peCmGblRKm3FAS22HFFPypn
cnMu/c3qREPI4gk3cype86SVuLbr04cdeeSOUIxdR8B3/AsYsEey1J5rFZo//cqc
uVBlFIlamJTtzZseM/GlCQJSioJMfUsLjyzIHalv77AMdz1seeuUBmKc5UtPMNrY
JFJ3xH/O8GCl/x0lCIpmXI47FrG2e1Beyoa+fBe/XcqQFobm/yg10kYY1HtI1Q5H
lgVq2QWH7EsNR1EG0MgwrqtlTWk1nP8CPEAClxosmQnqs2n/JArD05ENQk/FoN6c
BcOeU/anK+rjhtcRU7CBjUKQ+87nR+qhF/qIdMPKeBBjAAPyC3pa7vy7O2DCJrEl
UIfb3pr1QO8b60BOXLp/I8OkeQJMMKMtaga2fSeDV/o0rr9SicrS3LH3vUL4w559
xKSXkuBL4GNcLjlwOwJHvP3SFQTwLABuUCQ6s7E1MrL55ypBcgXq2kMlSwHf+F6m
o1dn+U3WWxPB7ZfjsndeI+qcyVPh8TLlvQB91JT9L5F+clGiMNcKCB5kgvelFfz7
tL3r1QPOzIyzZGDieLa2rqrRxrq5I9mT3St1S5UB4CZevLAeNRL0A7ujZtlwQ/Nb
gwkOgQH07q+ftvRjZZq6J44+zewuW7PqjJolsAerFmUuZPGLFXO+StEhmuTxtyXS
6wFASpPnQPQGsq7aIsBwKuIKLrCdQR8qiSLdKjRQVSPiDO/EzLvSVeFm5rHj9tFo
4sZppFfDVugGHa0CJd/yXe64Ml+Td3wbj8TyyIRO3JmA4+OXctcO26NT5vDSWoPY
W0Sr24Myfj+xLxdKJxTRtGMGNDwHLT3fhyAxIY+7UKSKjEbz11q5K1RhU2pCrTkO
qapvzvUH43HZKjhYUP4ltIdgROWlj9I2t2XvreyQPssP0geJH5SxAOa6WEzyVY3l
wuD2fCT+kXQaqWQe8qZv/G1MKXrM6fF6xI5jBYVftesbdDS3i/S1AzY0ns786SoD
YvzFpxAMZyD//8PkWaRsNaEwcBP46W9aafb3e+Yg/Z33ZYXPF45VgQq9Fvrcte5P
afLPI5oIJhDlvWSPCpBb60NUUdXBcaBa54da4T3HvNq0SQmQDvtkEAGbAKDs+nza
c/eN4/OFEpMogLrnk5PwbVcEE+EtkE/Ys4dmfB7voOla6o/fJfkEvQGLIijm8hhg
OtdU006XS29DIlC8ZsH2vLjAwJ/iqDkzrHid3F1tjPq1nd04hIvmk4hOdAxEiSPm
ocTwf7GIubytSZUKMQZdKhUtLEVRxuILjM3hG69xBu604rFgjpgw99PH62LSVH9X
/Gl5kY0RAs9C5HwWzjZLM6ZDsIN/xHiCZojj/3PEBxWhgMvbUzhbUKk4Z892Ws+l
UXlEVumvrtMIx59jxuuCxL17ioHiW00Ta3rYUMAm1+/si5CcmebQb3fizFMCKPXT
R5pIuPHjECiFjH8dD4mwEtU05vNEPRs2O8GNGVy33IZ73LkcXIDRc7GHQpTILAEs
Ky6ugTfWVOJMVfyah1XT1Pwp7pCu0HzQi1iuqSpRkOeJY4ILCvThUQpNf+M2hDDs
s00oahkhBrcRA8XICKCtU+7z/NktfeNeqTYPOKQsZCT92Kafp31kGQRzusihyV5y
xfbCxSHGP1zP22XqXlyPFKg7/ONvzYJrM1vcsVz3IMCJn+TC9r63A9OcosiggTC4
ezyebGcsrv4pwPeOh82u7X59ng2kXpU7chK+gsl8gjexMSEbAM5exQGBgzMb81ha
DPFqY+diUO/X5qfqufmBcOlKry/+cnXynpSgneuQ6qQiLQM8N9jzKC2jRjuqxlKT
zVPcT6tiMjNihh79kOqa+zn7hekAQNSpIA8ueJ9qD9RqD6XcTAh/VmSRKvHAukuZ
5uoT4o64bksCm9KpFcUozka/F8Rjjlv+WFWgoSwN+VIxBroQ+qJW56CzrFMiAtTo
t+sUyPUgkGhuh/vR2kaQQvcktf+yR4hFXg9g6LtK8V9Ktxln5g4MFfnY+X8E4Kox
qNGPlGEwq0qDaPJsO/sB//CLponLzOU4oihTcM9Mi9qkl2CtRpjB0Orf0+077+ya
UJVuaHYuJqN/dpAzhrd1zhCUmEDEZeh0rNyME88+Bl+EMn8vFquNeBb1R+FdY6FR
YKFNraQm6CH7dg3CZB0FALJ5NLv1Ac15/KoXfOUKLxLXyGa358C9jIOuLmxMbySz
nF/IL3D46QfpE2ZUuP35OeazQkd423gIkDV3goAsIf9Mdvry5R8yHCaT7c7SHYIM
NpRa7Tg0XPGExqIeOvCY23AKnsMRJUjvABcTP/JdZkGLLkNtj1xEZnNDVv4BOatV
pZX2Hbbc2Uj1vQIiQC2FnO4QM7jkVdkkH9R45HnNdSjbJVsA67kJe5Myz96zW5tF
IvnvbTqP6MokWIm619UAvO4k55F6CxLzsguvvW+ZTooqHoz97W+2yKV02jhkJy6U
3T7whoN9n/CeC/+/7qkDHorKXMCB29nvRIR8YPYojr5vsf3GzuxLlW+Lw7pspAL5
FIvzZIzWMIX9dR3OH+bTcL48b1dgLV7wjuFH/4jvXx7YP2ZIssVgug4XI6dEt0ed
Z4e6Pg28ckXrMtK2bcZi1N5AU0ahKiuSLkQ+WXD/5GEignY2m9mLUPwyGWsQv00l
Fi6P8yjVVpql0OVmlHvm6ga7qkBvf5MBFWMh1PxWEH6y50Z6UzLzqvbE7E/ogfBQ
FMyOhkV/lbOVtr4gBeE7RioF0S41dZ9qQ9BgSPAdWpNwgdWAx/YyySdNxiUCmm3/
8zI+W6p7Kn2Vo13bXKR/DCGeQHktwE8tNRsJgn0hMkFTodRYELbWndG7w8SNB1JY
V+LK6K3LSif+oAX8MqLTKJ/jv77IoKv9A+e1/dTw5URAulPSZ6lUXJu3h71YL+6q
RvUULo5eoJwn9KGOgv6JLLlbvN1ycez8nYC4QLtV33/xt81SFH1Od4hgFgSeze14
JPyDRVt9ZC96IX3lt4/8uFM2Jh9weUY2hMYGJJ5xXSq2AI8k6GOeM8gMDk9gTM8u
rcz/ADDwzkxajiFuHwtYhJfe4j8hi9vFv0j/4Np3MFZiVjHzylhIX6XhGBS5ku39
DOx0JRIzYPINznJ4PpFG+HZwh5YxWu1BCxy2NBRjyeUyiLaLI53r2ayk38CjRG36
KBsG19z1BZZQlZMVz2GB0zhTUcaX/Sc9R8Gydpt3wlfjFku02PWzb+8V4+Rkw50R
jWNVZSFwLnhelEouOEJ+sjFSoeMwOofFwLit8EoF6fvv77CxWu2TohUkWFIx4LM2
2Or0u+hF9SJdhykNkJ15jN3s0pOyFn5ruNvU2OuJ/FgrFBmMz/PyaSecuKlDjQHJ
d1qw1lFagQs2WtVs3rVTIpiVhqEmJVh0hs8qpo70fh4l6VfoD/sE2p76b0a+YHYh
h1r85RMb3geJr/dHSI5wZvpnjLAzmE0pQJDEQRSDoyYaon4ydkToqoisKJKCSvE2
mxmaikySc2rvMXiEzRi0ojNBorly1KTHPEkRfiZaR1teNYEPWDlfYJT1JfizGpcH
UZYLZZUMyJdvEmn+Pj2EArlsR6zXADmA1eTXINBiNPUqNdtvDj2aOAC7HSIrQunw
okxmgtCxgcJQi8GviKGnUcn+pI4zkxIoALzRC4weDk8keC6Q4rOPP5GiPDMi2gC4
6G2iI95W/BP6EVs3xsI3DuIP+DACp7i5EoSeZhiIiEPZhc9n9drSYUNWbwpjUx83
XzNExOQv/RLg1J/V3rWESYlUsF6XhD5RTWjhIee1hYVALKcQ5L1ofA2qSqw+uuZD
h4y/w/aVtuEGtwMzBOXsU1n4TABzXwhZvN06STi3FFGG3P7OLCNf3Uuhqm/I5YQ/
2/hbaCMVgzaGNhS8hGzMQnAlot8p2RvnKWPQh0H5c7VwD7wiDlrl+tz9BNDWjs4N
yIBeQamWFz3jv8upZFK++S71q1FBLT4q1FfWzeUgb597CcYgTfi5QVRyqeuKq2Ao
LKVJvNnnmqyGKeOcVoa9LRwdIo7DxBXtd/0if5IahGHXdBw807D8/j0LUY2F7H+P
M/RIxAGdFGplx6LMeigkJxquwAOSMy153Phyx5NOzMsmtna+LAaOsmaCAV5TaGln
oxQuB3xB7Elj+cCNAR6gnHp38leNXLNWbFFmtWNk8c1QX8OgyLcShF/SI2J+8Kjt
G9dp7a8/luCBSK7lbqQQqY1cFyLOpqGU8FXndVML1ZcPkjL4v871QM+E5ZjR5UeT
MvmVhjfyVpulcEJS1pPuVd+pw6rzahZ79X/nMfGzMhS5v+7Fw9ZSRSGkbuUXX2SJ
KOuPZieu6RkulzLYc/Gaasxlcrns5ac=
=qhAe
-----END PGP MESSAGE-----

-----BEGIN PGP MESSAGE-----
Version: GnuPG v1.4.10 (GNU/Linux)

hQIMA77xv53FwjH7ARAAiwgccEdY2usPi8fYhIUsrThafnxjGmly2aRld7OLVvWb
q1bWCavbm+q5jXDbPtS0yyKRkaBGYBswpu8315LKqLNWPeA0dCIHkXdaxdVU99cF
R7EajAU3muy0dgJiPTjMFLHUrenLgodAPuGMVOoa+OaVVF/w50++L7zHm5xIksKh
PQTv82pkV89TdmwoK5RzFDA6rhWAmSzPmS0sCbqnI6dp9PswU5isgdKOlJOmr+CW
TECbXTQHcLGxQTS3fKUoytDxAv8LbltnNdSg0C7W/NLNUt+hf4qCQ31FY83oAXvg
n4WXFu0GhV4CYQNFXc2G3/eSeGN8fzUN1kBJxKXOktQXjlH4rFti14g1UyUBpV1G
gwZ4GGz6kIW7aT7r5fDlR1ao7aGtkmm/bk8JNEhgTzYrJJLcfemWtV/tHRSkoa3v
5QTQ+Y6eMvd+XFa1jbtiM4FltoDPDzfzgJcmnkBs9xMSaMNp7JOysdPXsSYP4XPk
+9mD+GnF39f4cmm8XukMS6xykc8s/J9r4zJb0cpSSFoeiaDkmgI1lITwbxHK0v4g
CNYaWkobh5vHM9rTaMVn9pIexFEgfvwpVxAH0dX79MtidEXw8ewMPoFSJsS5yYGW
pXieuA9MQ2a+kSJPaxSLp/MOm9i4MiJbzc1yqeNIO5tW8fQliDa7OgFcWZy4WyrS
6wHvfJZ3sEwMLrprieAqM8uoRvtxIDgULZ0mA4m57Hyh8qrOgmVIb5AvWMkFXiQ/
gphnWKoacVTcqB3R01LBO81PlDyvLqyvpU0bUyF8Y39uQeNFjfRDemctnL+DnUe2
2XqjNHMahLtMxMV0dZQoiyUGrTNeImrdsv1l4rKeozs6uxk2f3FAGzkrb9ru+DzY
8gmtrgouWRlE1G7DzoJTsHUAho2NVgZ5arLAz9pXmN1AoAm7isK/wKOObCGFqgGC
BwoFK2hdbI7CoNW7VyTA8d0+rYtIhSSf1bRVl5dX860qofcjMkzHjYFNk/tOky4q
JljeZTHVcLE/AxHCoB1rEEZe5q9ZYdLAlUe7QPVQXKORxLPxmlS08OMJ9IgfPEWy
yQ2TZrpfXEUyK/s5HdJwzh7MlDo7TnlqNBSHoxgSzSKpWXt6UEOmjaiMIBhmi/DI
BofIDMy5M431MPiHVWsVCULq3yEF1USw+KpTBYMyGkjNDgj9YqMNKA64RifcpKSb
foPY4+c/QCYpWvvnxBYA4xN8Se9CMEMYadjqa8yOTvUvs4AW0RhFv+fuzLfbzYvh
CRJQ+8orn/XDh2b54eNWtY4buIHmBJLC0QFu/ECkvPmjUZGIzIkMl8CVOJZf3Rox
yOKkQP0g2EcKM9oGywbjfIK2SwB1OwTH7rEYD7gakpZM8I519Hop4PAluFrccczq
Xzq1WrbABmK+X9x9y/6T2QC1UXo6ShV+Jarbne4/dx2HF3qa/zCDR0rOAN2slH9Y
eP7NWQCFs8/HayXID6Rid0FiIjjOuh4AlZxl/Fna94Dycj3iPPPrCPoeR0aHvMF5
V+zlfRb8wjQkAMQmiepzMLdUw36Qp3AolaUhtqHs1LfB64gRColtj1gXKB5MAs8q
U3B7mRoT2ojszVlrN+HFRsVrnvaud80eHMRZqqffZt5IskqSvLJs6W3lBx4toJlB
rYM00T4ex3lF1hZVPCrDXWxdtHTLPIcJSsbMLYdxX+FCw2MughdNvRu15a+F4+Sw
ey/rHVs8i7sHnNsOYMXRSkNe5YgF296KncekMj88Hw/xFr5HgssSxv/zTC/805Yo
339ICnUZfANJsSzAs39ffdW/PkyT4fMeojG8c3oTfanLCJTrBn+S8r71JF6ZdJev
nsTluACNB+FTr9c/uhioko5WJkhtawBaIffkHzBxHOVzqfjt5j2eeLN75TtsHCfR
d4Rne64YSVUN8rvI02uCbRBxHlRunWYr41uRRXUPhsre9SR6sip+A35isYffMoOS
YtZ9yuOW7q2IHtaKGmLq2ovMTMYwHfd4XlY29iJk2NMsC1VYn2tYmIdioB2TdRSc
s/kOWkSEKqbeY/YMflh7wc0WKoxQn4oyLCxfZUKW5vNZek44lWN/MhNbPjsi8cPW
VA55RV9UvZwGxinWU72ALfnvy8EoIDJbIKuoLRGtSzN2FzvZOin0YgUvrdJ36Cni
MjFqtizU195kqndul9WxPvhWutgl2e6Bw2kAp034XxdFROYJnbpxZj1MkATV4Dbk
9noq+w250wGBdb9KOclMgx3Tv2MpM3EsrcM4vkbBfNfD1/OoS+rYOF0MGXIIDsn/
e+sJdCUefJHyDK5SkkRVrV9CrYBR9jicEBQC34D+cqAWV/YlJOwI2TUd85G0QVKQ
CVSDNoUGDhLrLEwx8hNNHEHI6zeAYt5xfsREwHxBijhmLoakOigUtdZi4r4p/EXG
xN8aeyKi0WXdZuSz3yE7OTqGgVZtZJpRIBJeVmJMS+Oy5yX5dHpwmcO+ZPs6NZa0
OX7wvNvrpjqCJtqpa+yDzjTbnxM4Ah6drQHXw8hkw1fxAVCs/sntNENs8GCDhz9F
p4Z/A+IYh5qbPjByN1VZTA8fjiz8Wuzkppinz+UVkNCyCO+4915pPEfr1ybMvI1h
13bTF2FWlCURv/LDu/EPLh5ncRhrjUG0YPqXpNsWw9EbPcSBJeJKZu5OGOeY/OMy
jnl1wEZ8H+W3I9/fp4XZxWzZkOjSktf29wH8ZwJK3eU7wSe4FXO88LdzT7Ozs26K
lEVMqhpSmClCeJUjza2FzqP/QBiQhyd0fJ3a2fgtyWUGkne+iwPEKf/Hh1uufBea
yLHxV4Z1zOub/2NJaZmDwi8oGS9mbeks2mQdJ1OqUh/Rc/9NkMSCWbgtbpcvdG0N
jKd+mC7ihISF6hLnp1ijbnXyRHUtPpPdqBpyhSJzh3htn9rditgp2IJYAN29zAfj
uiUihhuC/xcy1rDxPXYQPpjPXJyRloBTyIAJx4pXJM4yUhJSjdNj+Dgu/aQNEvNJ
Of5oqIrvbuGPfj9WmcSnVBhyfR/hK7ksAiep9C5Dc3NyZwqEbSNPNpZT6Ud21aOU
y/BarO6UxS4mmwj1+NHIk05IFSsmEr0eGd7/UK3bvJYxaLyZkA5jhsNfho235r8+
AQyMyaw3writVuotWX377FeqyJDu8oRAIVm57e5wkqkrusGDJidy7b+/HVHdcEFm
QVhW8ah0MRDcdWExs3rtUS3+3aUrD9pH9zXvgPVPBdeEi+6/tMqgc2qFpvTZBRdX
lBy+2gPHpFRIpMN+F4IVuKNzNVWW+zp9I6gF7i6TFlq6f/Pi2uDCVzptr32yqHv3
FR0XgdsbALKltIHEBIXepE7l8aAgs29bOByAWWV2KN36xoNvPb5f2/zymlefAhwW
cYvYd1u59PQBajOk0Iw62YpGmjSYAwkEOPdTQHI7IdQv6ZydvTRu5w4AtSFsGmOT
3/QTGYBRqFbR2GkPQurHkY1d/aqQtSQveJ1TyCXyul+GWRNT9qAhW7JpZucsvQnP
/+A0hGh0sku30dxyqNHIV0rNJgxbpe8m6EpxhpRYbAzlX9zQ7DxX9bE0jxq/9ThU
aee/6EN4xHwNApkfGxTwm+fOQ3yusvyHS8nWa58K21nfngqc8YWMtdCaIVMw4x52
TGW0zQPVtaBVMGqemBlxqmf9JlCH50Jrt59cJVW+A7W1UsdfDH+EUGpKYn3EJI6f
5q0qKVN1VU6bmtyLXclsitefxvlT8pwVkdIUoGIy3o1hivl8hZcI0Pok0E+7SXi2
HCu1eyvPWglvPCAmJvPi9UAWwe6D2zWib+gWppW77/ncqDGIQlO5KTjCb/9tKUJX
C2TFwBa58YLD3KDPDzVgjZbzg90zd1iJpMHylC4r75cunhS52wqqJKXIbr22ZGXz
HkYbY47ZhmoULqXeiQQu43tzIHOgKhLIU5x+I3HFt1hnzJPvHNfE9Dnsny2gNNNE
DoKasF8k2JsuVhjbtAlqdaWQCzYJ+pzQRgmc7v65q6wZuoTDlJJFJqhKyuWhEiVs
geaaTe8x875jMOs5QNHKOCnF2DyVJItibiiO5lRltyutoAF1uSNI678sDZ/je6WH
CN9+673jSh4v64guleGH4NOvwBJTRtMtaHOLNSM9Iws/u3QSdVOIDCEChCas9hw+
aU7sst1orGLGhj5z6C8MIy+EYWMIBw593TMEYpLrjGa7DsoFHCyiapfz5XuBBYQL
vox9NghB8DHZnflWK2QDlpDNHIDGDZnlZpbuwQhnjATilFIrd+4+viDeCIkqNggT
3HnzZo19nNollv4LzR8Cc/LauT7mcK/XIyow9emrRu5FTxqMRZhSDcdsoUeKG2Qm
1CS2x5GbDfgRL4rS4U8XrsQYysB9/ROx/Haz0oy2l0koSEmYe5I=
=Avuf
-----END PGP MESSAGE-----

-----BEGIN PGP MESSAGE-----
Version: GnuPG v1.4.10 (Darwin)

hQIMA77xv53FwjH7AQ//dkKVHFIsH16v+4li6BA1KOrx30FwUrxnMHum2lschsbg
EEz5OCHIvpGYEEvQsp6RbVnyqbb7xEMwISHZVB87+rIiXZb74RPCs5KTFykJlNm4
bnsP+n2PjpL0m+rdHvPD4MRoaQdGwd1RlM9oL/twLb9rUnf6udIvz9wzs+OhMupD
GRmU9bpzQg2SyRXR03JMGBuLIgMv8JLNFR3XXVlOmNYBoBKAt6etVen3SkGMTCcf
7eRQME06a1dnykkEy2JluhO2sOKtcYzwxYNA0Xng80fbbVFu0Da6qzD26UaqMmMe
XtPrieswUXJPNBIRlEgWnpRCZXGnu9Hggt/J63fu7s3/7GHTznrxkeZd5VGJz8w1
Lw/OdujKujRcHi5NwbyBuc7RYMcEbFU/v+bCr/9QuZMbt3AC+XYsUztvQq3z4hzq
pw/Mtm98k1fQPMUeAQk+zTpZhtkqYK6sI2k/65KA2lrZB9TFaGn3xqY5JImkCNZc
2EiNRlBi1kdWhDU3SD2ZJ74ZfOY5sinOg2H/2yJhQ1CbT50R/V08D+NCLiG1Ttb5
S3Jlygg1hrSIpqRlkAvhdgwhBIVBGWlrILkSwTDVVmxpW3DM5xqQz5rbZwr/X0Bb
5ImEGVEaOkfmfMlQnVjXqAeiVRDfwydSwtFtANe/LWVZ1oP1VsIPZiMmzwfiuJLS
7QGK8ACTSqMRl5YYiOVLLdX+kwQT+0n938mNVs0ukjy42wKRBbIc2kPKlD7mG7HO
MzDZKAAZzQpR/Dw8WYhbth8CTikoczLhwna3ocHUprnLH0mgNApBuGN5MGlaHUF8
RvoJPb7/Ws/fAlrOIa6Nn3wnuTmNlOXRoFpLpKGcyJhmQ0+0L0AW0UbmJpIAsAHS
LbGSGTCz/pYVuvOLkZrDJ/ONXFEYw9zJ9xfWLZdg+XQM0tID0m4405KaDXaVcsX8
NZ0vsWgWamBJdSKvEGVl1hLpSfnZwzuRCcLVIeT0m6OvC3mVWDADFYEyIZaXvlLu
L+ntZLiNeLyJsjdD/qCRM4uls3MZxEezN2CGsQWn+8IrsPNF8hvr7ew6xJtkfulT
AZXn9aOHv2KwWzoLY5UNyE2E2KDB/UDNgqJbcGEMXNMQkeWQ+7yyYapZAUMYgjh0
ig0zle3g1aToGsBxEOkmXyp2gGLdJGibLaPm0bk+aMruhwghkZndSCLwL1LyJ3+S
gSiL+2UNUNPjLxlX/gxAum7QF87Oglkx3B1sGSjiIqHUQn+1OlJXTiJdE4TmSrAb
u6oA7ix6QeValQPTEBlu4q5nSCjp2d6zx1E2S1VRfHLdrD3xom/iIkYbgT+fRPHV
CNIUQc+CSyZA2IkVQSlAYCtMfLydOnMzrOOBgQT1+9tZvkkkli9LyteCPxiKSAzN
nDCYBKreeQUYQ1BTE7N2dEhOb6f+vGqMNg+bMtkoiriniesrlG3qV+DV97YJ07eY
PsZlB3BjqHfUcYTqyMRpFQX9qX6hwN3ih04SwdkNdA7pP87phDQrtL8QungVVWWN
LzScsaeDs3LG871Gz9W8IJIWmPECVGhX/9aU8Tj3AnGu6PWplj2QrCaP4O219aQu
pbWUp0FglQVlVOiSUuaOwmdXTaFFPW9ocvBQW056igOrcWZXM61XqEOGjIZnGvsW
o9v6lEqLCtnNKXiSDw/+uDoMhML7V3crtuOq0Q8xoYipxzXW9Y7lHQmRqHxfOYaP
+LlUaWU1l6xqfZKzQZXoUZkTO4HdfLuuAoyBN6dztJOlxi7efCt1AFQrbVWtCBJ7
RIR2rVeNMe6MxrbfaaZEErK/PSe+2Y+EqqVd2+3vKDU2u2WeLoCXdeayBMmbBF27
6UIUIGMFmC7ylAU6bGs8N6Zb00zMl24eIeXv7WTnnldLlyyUI8lOKwGm8WKOyyO0
rVbIzNbTamoM3JEKmXj+QouHNN7xZUlVroDkOrCINpCQlNrZw5ksa4NH4C23F6tz
2LMtvf640T7qO4FIvnZItvCsTsYIH5EV47ouTtGRDe3McZPkpqUtlD9dtnTqupmN
3eF8+o2lYfMS7D4JGFIua35g4EphlTzjkQ49Yp8PBg61n7zLiUKpcKBPDDOAiLsD
Aaj8/SdeouMXqm9inMXRTfi9OsNVERD9UN81xZcKlVJC9NnZKWtW3m+uCy//yLV0
aSntVbbrMNlNQ2OzbztAjBeuSJxQUhKtJp5RPBkJeKH56SloCEJeBgB9mmXR3sA9
o+5NY3DarVKs0CIvda8mB9L+/uPXYzKRnNOCloCFYBfJo00dj2DdT01o9UrkTkXH
Vd68HsKWY5Zv4qo1bvuENdhMCKWhzZjVcX+Ra53G01qv+c2mB0I1Y3aUoz22fZYd
W4Szv0VNLDOTXkTwrgMItasSt+gsEQf4Zh1LZncFjvak8JTbs/44BnrzU7x50+dD
rE7M4dikCKomzuUbvRd60kDWRY2nEbPUoTHqDgQSjDpMqke3TpNQez72x7kda5hU
aW9s60dyzeUd/U7rrqTkHk5qlbTmoOpFmZU7b1PmGD9Whw1F32uF3f0oOtUNG8mb
dbweLcZouVyhJksiV68KNgAlrxKhJaABSFynBhTdjcJwChip0hT0AOe22E/B9yNj
ZYVh2OS87eFLlSevj7Df7kLzhjWSDU6mtn+av7kgCudRC6o1mmzj4j0Ji88PhyUl
9jWTOJSnR4/G3mJCSyuFw+Y0XMlEX4abN0ey2KJNw4tdke4jKKRuI4m2C3hjplKh
NxnbO4Pf6OoCQGUdJd+bSBGlF47NE+1K6wjAPGTk8YiMMMW6DD+8cmxg7f8RBFfA
tkwKfhnZ1na5UTbX5/WJZaKq8YGIIyhjku90oDLDEhWmryh9N4OTWgE9K5O5HTCs
Ot5Ewvc9IYcwvLvSkPEvsyXF6sQ12R5dKPUEJEvorXmKCHwLXAGnNgyX5nXP3BDH
I+uwwi0b8+jYWA65trKMoSjxqbe6aRXMxpz1gI3e6T1+Y0lJjnMSZQg+QxWr1XzM
24vSCfSeTLMT2OpNd4lnX+XgB6KggXWAsUzFqIB4yl5bZKCQdQ6R6xmlQGhD6mm8
szx6Sb+42tf0/WTfky07xJ8y52v1al4M4MoJDcAVZnE+r2jn/w7JwMe//IDg4Gkf
X8rXjM511T+rm3aosTv+IwNeorlcsZXiqXhwFULcOMU5otUdqR1W78+FsARXOX6U
CNkwy9SDop7bkThSvel1t79J8JlbusvCgvd0It5cnoiWRHYc8PA2LYboKVcJyzDX
5bnfzOG/dhXkDCyzBeovmajsWNkGCOuLKOhrmVXuQQlT6JQVv8SkpY6flS/mUXyL
iSVyZWm4+T+Ozw3q7WB/NnYT5fS2DJVRnrXfN0jUaX/rOiiiWzBC4YcwV2TQWdHy
40JF6a2+jPa3UXh6MaFIp3Z1nP4gVw3esCkj7/NJZRcDNLkymIYFIhX98SiBMG8z
7LIA4u7ne1O/2RgYy+DVvJ4dv1ib1a90mByTqJ+RR3+0f6Da4BGsjaNe5ZIeQHqA
x3JeRBDVpvxzGWDq3VN44VSZN2ezK7fH9Ua/6sW4pBgdgHVq7XBwwWdq42nkAtia
q566TuwkST0berH8W7kNMYuctDOU8lhG+ye9uPv8zxc7TEerRpop17nBhJh2o9xI
eOWYm9vqkWHaD8cSOZ/nTayyIF7hHTsj72mp7Lr7F+R1o4vGVkU7l78/X4e84LPE
bZy2YkLnTK59FUjE2AM87wv2ZA75JZ88mD3B44pfTybzg6Rpl2ofEhP/6dRKzvD6
xkCnGSYwuqmrY85ImsivCPic62uexMbazfmgiH1vZE2dali8bHCkMOo06FHmwjt1
o6eitiExKYDFg8lqHVc9L1C3ip8qRLwHpxm7Lfg6+fs+IiyrTQAAI04B8IpyQCi6
qG1xQj1jTt+s6Hv1k0h2LGAI+1UytmxENHlgcEligXO90oFQ/gAiSjrlFlE7jOi9
y/diNcVVvDky817M4klknq/xjuz2vdzCFakcSL/B8NbbY8+ulJitiXqFmvJSsNuK
88fpl1Kla/M6OrvNxiM8o7mDX5vbg1QekwniBVsjBNMKnNG/P/Ko47R+xwMjzPf1
paUdHOh4gO1HSqsR1S77N9wBEBMUFV7UQkMMFX0XyvMUOxx6Hsvku9Ha0pvPimp8
IytCLkchxzsqmN0Yurkafh+210tP1fWdF9OYC9lYITYiEvw6foFmx3no4A1hFm+v
H1i0ajsqgfEdkCfPRYcTtAYdynjLOYCD0FoZZNFtbSGuXDCYrGsYis7tV+s7Nr3C
ecF8LGTjjjxxeoZLwuXWhp74t+5lCOv5e0NXiA9xB/igMfZbylgs3YOK+YDdRcmS
NfQ7Ca1epcvLjMKwMRpN1o5LTIzal2g95bmCAncnXthV+3wwrQf8FBV58ESiAO5w
07FQcqL7iYHVVBUkNQI3NRHIb58ip9DyyfCLjQgpNXUwYNAfcb9R0OSojJIZOy61
KTI3Z90FCR0jOrsf0/3A6ZvwhoUNz9m8l9S4a4hxo6FmeHyxDClGTbshaVdLXvLY
/0jFtFsaegTMFH73LOF2lfxpWIcnSTYyZmELNJhDCogCT7j2JHfgHhb8q0HhpgDJ
JN8WlR6xogNsFKAIvSpBTEBiGiKiqDtZhvMn2VBgSC5K0KBboZvEf6cvHhTJWW++
dFuO/Wa3NE8zzj6e3zPjyHefDC01djULmYu90PmOeWRnoN5sWGRohBKrhk9WhDoW
b/BmzpSU97KjIZsI/DW8WckoN9iUCKzrFzZR0piFQkbORZq+bbkm5xE7JjVT9hfc
N5rCOTsRMptQlwBRJDh2kvozuJuhs+lfbu3xLmdHTSr9D5fnwM8mTtI9xHIcLWbg
EoMVMOGWL2eahlkeKKsTOXgj3FxDHfqknrjnW0D+r22SNa81FSdjuAAOzgA65J9a
Ed+GfZ84BFK4wdoU09bRYxjGzfKDZFdlljfKSO3tTy0VIql015hiIl3GJe5qocV4
TY5mCD/x/untO+6VGhVgOFBUkTjFg9ySXzVIav/YhvAHoV1CD10nyp2Xjc/Q5xpU
W68E3Wd0JjpxLrnhPNOlzPkKqCspyd5WNl7bMiw6j8eardvvgv3c35wP7YBIMAwN
YbzgyvV0HiYR1e17mGFvUQ6HGs6dllsvpgej7l0eTklKAVJf9QfwZG0/H3uszZFs
IW9uc0pqRTdX8Dz/lHNzltZmMxDU9/rQA0RQu5j1Ft9K+NKV1bI3EEVFamyo0axy
82xkP1gPXf6w61VC0LKAfFMiX3YdKKmE5vkHCv/TprfHKUYPHMglXX0ZYtW2mlug
7smMMD5yaP50fhqGhdjgn1AjVO1WPsN+ZfgCt7bzOk9cwnVW1fbSIvFKTTWSa2Mz
RJ4b8vL0p5lJlbt5qVWpOpXZa3+0tO4amgbGKV6y87tCmPb5aEHvPjqhGjDu/JKR
RAMkwdCqKtKpkQkwYCIUkHUxpmuv7XEHFnGgnIt1BWXAbJniibPJqQRQbB+nbUlo
u2Nxp1WkIbPSwyQC3eRrPUwmjp4MpLM080G97cmi7hUn7cKteICOmw4CLAOYzex/
j/VT1KFxUyFkdtLg0gr9pANX8pnjq9bLCCrFLZfejdiS+AojlaVsW8PVQAvXiBui
NNYIOrM4bal9AJOCOfKB5+NDkK3wPdQU984qdoJCqbSXtNh/lLmh7BAO1kUGAIaW
fC1slVkXINZo/sQulJCpp9qZXCt63SeJ+FMFSGbsNdLdI1rqNFXM4StND7t27Ma5
pyxbABLu65eh5wm3mQ5dAzOOMVWAk51a5vwYaTlcNqL2/5qZZ8R+JXHUb6qq1KtG
X0luhCe6BzHldwAKK6WvuK9xpwQSmDoQnyywj5+SemGI0p2bJxb7ttbtB6RNBAVC
+vfzeplU6yJT8LUMe6ik6G5e2DO9wbm4Zg5HjOuFFC6j5ju5X4hjOGDIljZ1h30L
BqmhXU8DQyFJwlae0XR8H3FNijnw7st+GFlvARVq00LP03kuDB7RHhLXUjfZBcE2
nV5e1CjPp9oRufa1CXkl07NRKwuFyJ3gf9SQRZi/XCw83yv7/VCdMoemeM3urx3Z
x7uE4Qur5u/vfZca4fedkkyrKZUdjEGJTNNBsyH5C1NIeH9twnZt31rIacquSFyH
kRASJZlYavxazZf40bzTJ3PzCe2bg0/ARFXsG3BwSTrhnWLihfhQ1GvKkYe0htll
4tXUf/Bi5ENwrZi6grvCEMHoHqDC+LLFs/zDewVK92kXyi3p+uIgKKZw7Q0jUjDM
/CMcJu7eZnS7s1TxelA+5i2G0q9zeC+0gVl4zzQBuM/d0uxT+RAQl/Uy9X+lylZ3
naClueti3/4/RkXjqUSeoNn3vOqppnXZ/EWDfz4AYhv0gBC18rdEiQPZ/VvLz8FJ
+X9DYRsHxi2I9OAsjm7PvlSsjs1dKR2HzmZyqlqGbNx4PI9NNXOzWvbS0OBCXj2b
wnqF0gpRUaCtrbzXj7l4uobNf9aLZ22QI1PCqE6j7JcVGyceawxwlcPFlvCUqBLb
QUiOffNu70lRuFgoj83iP698XZ0/4v/siy8U0prifiMc1Nw+iNZgegldAS8ECyqJ
IdwtyAoGe4XNJlzmR2tIhy9VDwzLAPD30sk21r+yRNpK0Bm223bIAQhJdHV9b0yR
rq2wWoLPWo3xACjAtbIHBQIgQ8y2y2xIeHGXSzjtLvOYYeSEqrTr/8sLElEcHdn0
nh5h03FHLDaw0ysppHceU9vtTJygmHcv8MhjgnXXt22WQ+GLXnV6svFdtl8OFzp/
vDB+9sJ2iqgRELT4wvZAUAZ3c6+SEs/7KlB/2noxYLps+C5CQdupuIEtxPCAcgVd
5OVgWN62UExJvvN8/Tj+9meKky0xLCUYtYmKRazWRQqQPxR6Soje+f++ZV653wun
6dHFzVRqy67wXp2+ECM75hz1bvho8BeFeLjlv3UxHQy+HZ8qL70FO50Ot65z1zy5
M3n5e3jHnLFO/gY9yaLnHuq6kvzqBlsmEnRrzlD6E7tH+kB6sovUdANfPhcc3hAm
/0+A5yrWlqMBIxsOEms2BcJqQ8dyPqLqs+tuaE1tR45toM4NJ7QKHhgCvU2+iJsc
MdTwOLlNcNXvbTaDzfLelyFixSdWh1C/f/QEhRyLtCMGd+lye9n3lHbZ7+NpqmwW
/beNca1/kKvgrWeuHJEtm+E7L/cP11AIiiKgNlMxmmfHrWOLVYOFbjSMix5zmHUM
Jhv32yYVOGej4nE8OoYyD1oRcAPJzUeO0dRk/8RoMfF0JaJ4r9c5yKPv5eMxWO2j
2PBBV1u2QhWEQmheJ/89vQXH4MCi4UP/KQh1/yaUFuiEKiP6EqpS2QX8UTsCyOYd
F/wazQogU5QN/857/PdwJWHjzI3ndGAnre55Zv3a6kyZ3ftrl5cBkSEe45I6QUfp
soDTyqgm46BJyOd3RiSETXVqoOA2zgLylIh+/jhcVQXGabiWMicy85fTf5cz1Syj
1DLWlbcg0OS+V7XHgoVytEEiC8E0uxeHYk19MaU/u1P32F56eVzhQJAEcUaPNRMM
/oMKjyH+QihMoNWG4EI6jdl50eYSVxmmx5hijOqfvRDARdK7hBhA59b/H/LWxCQ/
zdliLMRbvs+tC9E6oAuwXaTqwKUyV4hSWsKxL9pd/jS2TpF+ldipo2p//DJUOiHr
oGaZtV7sQgnmGvD/4HEgUCiTxkCeu7QK7wm0h+FWWGhRAiyUpDLPJ9+fFD8ka6NJ
TnhUS3NwvnZLkAfnXu9hfdXw6tkW+gNdvvxYqOo1YLMbpGvVzgFLVUOtglZlum3w
ZYnTWrkjW50Nebehat4FFsCJV5c6L+wjbe9ixPJ9gSkO6qLnvnY0P6+Wl7TxXM50
V7Yglq3FPlULDgEdPzu9T8tBhDk46Yi4hu800q5Pl7mdiJqI89WHus5PKNTC3UCY
IEzv3EKgv6uVRk+Y+dtlLQ/4xOJgJkCgJ3RdjDXUIYpEn21gO6PG+isl6RYe3uZM
tQOxm0H79ltR7ZM/NTNb4TGyJABglCy5WcdO3t0iyojfEzjzWJU38a57c9DIXuPP
uaMD7oviGrEMC5HqfV4E/oHCKU+jTCpbu4/J62BY3Or5hdcD84xM9uwOSeVAz0DI
l8EXvG2JUBl18rDi1Y/zvOMjHUWZ7dwFmVIu+qNJGcMMx6sAqa5pRUcDPSjhDUh+
4pQ2fK88+QPaXg8qBXwM5bxjYcQT4PXoyKy6wYjK8cdouABf8pf4opj/Pq3V8KKf
Ov6COouJ056S3OdbHx3eXD4zgtghcA5UaYrD1DkxCcOP+utZjktFSSxriBUWJQsU
fQc8Khme/S1BXv6jSW22a5iUvZcaz1Cf6EF96wlzyZdDafidNf3p5sbUTXDt+cZ9
0Lan0dqT9hfTifQcmEzT0dsZ5NFRQ4wDgrQ5OPyCDXx3+8a0ezoAB5wxCyqWdtwa
4ZkzzGdnH742e3H094wM8VX4e1TjHFRvrCkPGxdpFIrrFmwx57ji92KJyvya2Gjm
8hhFP3NNzQ0wvD68EGQP31moEZCLxio5954yp62ka5KqZKEShWIar6VaYGPmAxjs
bPRujjt2fsRcC6RC6O41ptJUVsctaO5juKAhJHcz5bOisflXPFoSLC7qrf3LpHOX
pe6eYK0am942zjcgDTVJmGqYcNg/RVlg6DuaogDEMr5EpCOvGBLJdfWy9IztVV1z
xIVi0osNG6uPq/Dlpv7r2jABx5JlgjU3Sh3AA/F77DEFMWRs6z03fDFrCQRqdVGA
tCyegg5AwBWNel6kpoh8V9RllYUHmKVz30Uf9+ydFjcYBL8BV2slmQQtJFeYJzHO
iBMSFENcoWiVQc9cHF9i01k7/wxZvGno0ZP0Zv1eW98kZ7W2c2DKJtURINwPo6Uj
Lb3eXhf8K+nxKS4F3XCnlooD3O8mWrmayA7fVXt/c3xM2TWx3ignXLQS5zhtVgYp
v4jXqo38cLTodCLBGZaZWPumOix8KI4xEjAi+vA8/rgKZrG5ZLrmnYAaSuBSuzRz
7SyrMly+xul57dydMEwaWRoaG5F0TgjSUzdpT+5W76F+uEvzuUN04hf4RhbfK3qJ
jOwYjGNQxUJeIU0oie5do4vU/L721MQ1DX5dRCH6vbCspNKVRpoPVLK4Bd7nY9k8
K3eQLJkiFkpnn00YcLiWjXXfKyUG+isgTq+R87rsJD4VXy2wZka06giGe/fzVHWm
LsOUmsxft1NSb7viAachtSCHFb/1Doy/AmDfs0TA3hWV8qvGXfNRIAg+umzBZxKa
2CbMZCOQujx699Il0J4fJWxuuf3xqazD7WhON1D5cXDHnqTtPz+Vo2gUFfnt3Xj9
zaKMREHZl59leO3WaRUWpZlRVn6jTD9+m4oxmxEE+lmzwfWVXswmFF8vvrD5u596
vamkYJxxHdWJu4xXodREcHCLauPNWg8DgAY/PF5n8gDPZ/ZC+hIbChQK6WNNDGDL
JOH3U1McT6/ZR8lkgBwgj30W0VRv0gmDEQMksNn5S8H1DIX+IDwLiOamZsIXvXzk
8tdPYb+lQ9NPi3lAyh/bcfM787aztb5NO3yEgh8gY1TbXhfQS3azr9Ddsit4O5ua
olBMyy/MrNCyce+aHE1Zkv52zjulNs56pxrfhJq81+dF8OiadYV6iTzatomf2LeZ
etd8MBFEha15eLyv2jQZujCY9AYqYXUoy6nOe4MWhqqg3+cS4f7LLwsmPpZRUiVI
lQjVXglWzLC8Brx88v/qHnAOBqa02IJ2BX/zdVFcxOE9q6pA1jxdyD9lcKG6z4gn
WUOjTNqKWfWXdg8peWwjS+VXapga/GR+4LgwlTHMr16hZQVzOH5xbQ9V0A6Um1Kq
4n7aV1yeJgoK3QIVKGQEmAA2vYqpfwT/+keN6GA/EEeNlJPy4OPyiDOuMzW99Ze1
GV6a8SOsE/SfII1vtOhYvyrNMvRe8mxO5H/Eo01JuCjmzuTrw9lhj6osZFiZYQ9Y
FE8vldRgCGK4Ohpw46/eajQaJQkSck2NSslxQPu+t6odTPIDaHlR6ZKa1jQu68TF
JI7g9+VxgNPfZzAVNXOf/srLHP3DaGC6BlEn9KGfA90u3tJUS2lWj+8n3SGjwHqf
NESOghaRs7MeX/I4ihykIHseIf0UdLX1Y/VLzqqyeucPhLxkdcxNDSVyQaFV/wL6
vq8XzpcH37v0ze5Ni7LuG7dmLZccTHE19eNtr3az9ViVay3drY2OgZxzhmBX+3w1
g56DcYyXh3HZ31NAcdH4QvD2gUev7VYigjjohSFcmI5hyX4bXHUSHwtfcBtRMAC7
dPaY6XJ27By2+er/B/sCJ11AeUAQXNemRTlVITlmKpkDLiF8EPiMmxF40gf5qCBN
1HBqnYUtndFOTBhx2XBZDejmsOuCeziyGRtPf31UR1WFbYZ9/XAZ/L3h9w9mSfSY
VLMw3fZksYJa1dpzYY/21qrQzB4vjgfYw6s8GLk6les+Do9n8K1MeuI1lFfc2BWx
D+8m4mnRhxcGZmLOl8Oeh7jugH55ULguL/BwFb9FFuaOZZGwWO4MEIU6dClDCfkK
qzD/kZ8NP9cC3XKT6JDG6VkQx1gtg+KA3XlJkeAZ98f3k8Qf1niTpp31knr1Gpx2
eMuVot3vFZ/E80xNvAjMtij4YJ25Dg4IIlyaNgxvwL3BAFcl2P/fNACAGV9mbQ+E
dXaK4dQ0a8jNxFD+PLcfOk6O2dZbFGH0LqpH/fzwphIHIhZulhqXJLsrDyRPvDRI
CtCWN6Alr7SQGFGZK1QbsMh62KPtYpEpYDTS/DMYHsQxzVXG8NzqL9svHfpCV2c2
yJPQna9/snGOKnx1gk5uduKRAHV162I6XAgx9VuO8f8AMNitN7dWOl735FAfEyBo
xJLBH34nswcu5atq1QhWcVBIZq46aDAi3a/56RQ21FoVD26tZvVaEjk63bdU/Zef
e1PXqUrLwAMJ4voWRUUcJPl0QdodFrKYg1cAauV7XN6OR4JQtRKBDwc5ykjSDuXJ
q8Ne2xBqnsJm8G7GPmH1eC1Eb83YEjyxhoZAJjd75RZ3vVF12m83Jq+XvsCHjKzp
r5A+miwG2FGtl9AQ21ceoD3ZYLfl3dAOmPxTJy2Yza13abtin0Ffl9TBfP3Dqe3X
tJgoyD5mS3F2HHMNzU93GuxT+eKKKHVcyF+sf8dRmiG8Q429cEFM2HcQIwKKGG0p
VN8nFkg68Cgl1gVeTWkc/aYsOqWHjMqjC8Mk/rJIW+j6xF0ubNjC+Qx2GrXMxpXM
4tn4Kr3RL/+OXagS/1+WadoQS71WVvEffF54RaSyYLHY0HiXXs60/Qx1MwV5TcgM
+pMfw8D0+nVICNkp1liHd9AYCsiMvcLKxeYnl4LUwXXi2qINtr3Cx6oaDjxX/1+G
Dqi1L/y/twXhUoLZgoFlmC8nhWHxi2/x9DWnKOt9OLWFRJFl7kH1g2GfhXibe4HR
TNAKOL2gCfliVnfZ51ntS+ljXM3y6v/5rUFZuuwvD0h7eftokzspkqzJf2AG2WOt
rm1S1MXYB1TcR9N/qe1wvPlLBIay+MKurMd3s+kNoit4u9ktZCMMpRBJSk/pCbcL
Emp43xyROCivH3DrJCMLD2073PSLg1+oVqf8C/pXEges6fqB98jJC9femn6N/plE
uaOHZ0/8HDps2/FvuvQUu25PkrXcO8qms1REM8a7TBEW7kK9va2F3m7MgH9aGXtD
yTtikyZL09SwZ+oBS3ELWBggqJFF1K7+tbrhv3agqbBIlKQ5ggFw1fd801VNi/3W
fkW2hEWriaImh1MHynhhmnuajEUw5Ryy190U66qdbqxXQnvcqMe4d29YTCDNSHHZ
nTJJXfJ7hMLos2MJv8mEBzoH7PNV6wHRZO4+eiw+mCgSf7N98wwWJuX3yuh35Wc0
MEP+vt4LzrIT3GGb93xXylOrwBHx/nA2PCCicPhSJZl1/j1p4llNT94DZXof8o1B
1BrHUapiQD7UUlHH/iNXLZQnQVNTQsQv9ghh3JwmS4D8r6WTID8uXuVJkWeTOjft
GopDRB/3BEP5vusZj0a5QbburAKjs3igp0Ktw+j7k6CGTpIZOi2jtsY3zRXJ4kXj
d1H1gxWTCmBOIsTkzxgVm0TC7rg408pceZbEGs8tkpP6H6XZ9WeGGPsFf3gtqniz
obg5Hd4E8qKdC4wlsUJnHHCcB5rZcYXlPgBqiEp6RtGklker7K8xNmgBg5bypcn0
HvEcnqK5/RI0+DGD+p4tk2oZZRJ77EZ1/ydV1aXHONRAXmFQ9Ux02eVXM2qt5f+D
c4C1JKX5B0FDn6DDNR8XpUtojCapbHfON5zQm2C+id8FkFTGy87JnMWLdobZFfRG
Jqe2Ha3PNw04j59ua41Y2d1+UyZ/PnlKb4o9PZyTSKux5qzHIEzlIp0VFpH2Wt/V
zt2NjZctoUxifVHmfSZUtwN/LeToBzy8NxnfXFc71AjoRNIAfxC/q3A8HZGrqlxY
bjFSFnjKl09yTC7KQLid
=E4Hg
-----END PGP MESSAGE-----

\section{Global Headline Adoption} % (fold)
\label{sec:global_headline_adoption}

\textbf{Need to bring this section closer to the others}

At this point, {\tt Social News} has already filtered local headlines from the initial submissions. These headlines will
now compete at a global level to adopt the final ones. In this section we model the global headline adoption step.

Note that before beginning this step, the two properties mentioned in section~\ref{sub:formalization}: unicity of news
submission and non inclusion of groups should be checked. However, this discussion is out of the scope of this paper,
since we do not tackle model checking in this paper.

In order to provide the reader with a better insight of what is concurrently taking place in the various groups we change
our specification formalism from $\pi-calculus$ to \emph{P/T nets}. While $\pi-calculus$ proved to be very efficient at
the local level, it becomes cumbersome at a global level. Besides, some of the requirements at the global level are more
difficult to capture in $\pi-calculus$. For example, a group should not vote for its own news. In order to express such a
property in $\pi-calculus$ we need to introduce data annotation on top of $\pi-calculus$. Finally, previous research has
established a close connection between both formalisms. Indeed, Petri nets are often used to provide a semantics of
$\pi-calculus$ expressions (e.g., see~\cite{Devillers-Klaudel-Koutny:06}). In pursuit of simplicity, we have resorted to
use a \emph{P/T net} to model the global headline adoption.

At the beginning of this step, each group has its set of local headlines generated from the previous step. A group can
then publish its headlines to the \emph{wall} or try to form a coalition. Generally, a coalition is a group formed in
order to increase the individual or overall utility function of its members. In our case, it increases the members'
chances to pass news as global headlines. A coalition, when formed, should combine all the small groups which took part
into a bigger one. When the coalition is successful, all the groups which joined hand can now published their new set of
headlines. When the coalition fails, each group returns back to its local headlines. Then, the group can either publish
its headlines or try another coalition.

Once a group has published its headline, other groups can vote for them. In order to vote, a group must first execute
silent transitions ($\tau$) and then select news published by another group and vote for them. All the votes are
collected in a single place, \emph{global HL}. Figure~\ref{fig:sn-news-Petri} depicts our P/T representation of the
global headline adoption involving three groups: $g^{\nu_\imath}$, $g^{\nu_\jmath}$ and $g^{\nu_k}$. The figure shows
both the case where groups vote individually and form a coalition ($g^{\nu_\imath}$ and $g^{\nu_k}$).

\begin{figure}
	\centering 
	\includegraphics[width=.85\textwidth]{socnet-Petri} 
	\caption{Global Headline Adoption}
	\label{fig:sn-news-Petri} 
\end{figure}

With the P/T net depicted in figure~\ref{fig:sn-news-Petri}, we can now check whether the goal defined in
equation~\ref{eq:appgoal} is achievable. In order to do so, we simply have to check if from the initial marking $M_0$,
where each group has its local headlines, we can reach a marking $M$ where exactly $N$\footnote{Note that because of
silent transitions for voting we should deduct as many tokens which transited by the \emph{vote} places from the tokens
in \emph{global HL} to obtain $N$.} news are in \emph{global HL}. The advantage in checking that property, is for example
to quickly alert the different groups when the local headlines are not enough to reach the overall goal.

% section global_headline_adoption (end)
-----BEGIN PGP MESSAGE-----
Version: GnuPG v1.4.10 (Darwin)

hQIMA77xv53FwjH7AQ//Ws4xxMMCHvOF0bx66gD2Z1DRSD48b4GPIbltycBTe9QV
Ej8+JyvCjlnxRVWWH4r98towdVAIXxmRMNNcol44Y6CFvd48E3inobR1KmRzxe9o
hgukb3U9Fs74D3mdXggK75rDmttxZHizpYsbeLlAIJIytttjibehItBrX8PWU2kc
6sbemzBP4kgqmJZ02kFVUs+R+mbhMJk85q/D5zRVn2HhwSpV/ppIIh5F9Xk8yLTs
YcZGsbdoeiZErtZrbdV7N/iv3fUMf0vVTtRARuesZeMlVcqI4QJ6y+BPabH9iWyU
e+v/YsSgek+aBuFcOsLXgiysoqE29gDhxA5taa99ctvxuQblCeMUkzXDx7iGPfHV
gbTTd9u6jMjIdFopM/zTn4S6pSYZcmSYtltQOUOcxi1HYkNqneOKS3M0u/OhP7L7
Dp8DFeEcF1Al2Nc8ILJiXlAUJFfQXHEvOsKpD/C3HcI/gaf6v6OE4G4SgQ9bJW0e
JMLMIE1+cD5dlusdkQexfbj4RlAkZGVkUr0HRlEHsZluvUNmQ8aFGbWlcBLJtrqG
HDyQ50MPWiPQGL0Ry9PkBslsP6nwjBHDXCH5gxFKiR2A6ul/73YTCx6E6xOKuWwJ
JaepSqxqWNP1bEBmjkiLMoSp7sDBn6uR3v10U7uBCQI3j7zZIwLkOblIGRSL74LS
6gFkO2HQVwQNwALEDI8rElodPc6RQGn5tHtO53Z9vIP4GBZopVXtMVs9KPoBxmMt
0IHupoPj4SFXq64cxOEdkESHDhXfpxYm//5aFdbBPk2EOMkhoKG+dYxayNV8ZzYE
KF9Hynv0cZhrJPJr5i6GuqKuzUfdpawpmOtGzktwEQ++X4GC29h/qMGK0GYZn3+p
OHxVTkXt1Zbe5+b72R2jAAoIPcX/iLYKtaugImFuPSU9m2GNnfA3NDkXnzZwqARf
YZWXLiQq0oISm/3AfXl96mFT6tl9pZ85cbSJRWdd2x4r1aw0wqpWtb19DIkVE6P3
plK/lqxKmzsZPYvgP3GeO8XEx2IHBcnY4ClzORJ6kdmo9VueGONavQjKQanjo2dQ
GklXBJKrdXzzZ9vupCUTq6U6Kp+3Dfzu+JaKvoGVX6uuE/VgNEBnjnlb7RK5JBom
MPsDR0SdaKwOLA3p25GMCMPT//ZkklATcMGCdXPnNnOVtP0c/RgrWDW6ydYKVfTD
N20HdkPztJx/EiO/FkuEHhoLbFuposTPvFFccndQC0QxnLN8QZyrmAgbUbSPC1WZ
lK3AZ8G5Rq/+mJZznRn6NaMxDUlc2zl45BfBrjiUzkJXYGP47AP0aVXAAAsQizvo
umYC0fASXXAW3gDz1qxlCxnZKZJ8FSWcfbusFVfM/G3bJ4urBHRK5FOq1fKepChn
c5p7O6eO3D/QmRUI53rAzqjSSbBV2wGQ6byW6XKc1v+Uz+UCQxyTswLPCgDwu4sL
BfGuU8Mw451D/ITeKh5753Pov/yOciZyHPq0iSsU1aKIur9GIdS0fm7TcodwEDEd
xVQQrmVbHUr6bYaKwcwfXu8aNB3eIEbtQS0t/ZnwY69CW/lt07TWwuNTh0YJxXqu
vBaMPpCcK6oOP4+mLKRXpa1Jk618gmh8zaGKK/SjUxxA4aMYcX8XEK6kUyPc3Yde
SMo8HCSp2dZYZRPjlY4u9ygOP410Q85QJDMbYQOsIi4D7jLmUIG8bLRU657wYQaN
jDgEsKev2Dhtp6NMe1Q4dj0xVChFxZy+NntqxMKpIdLNnJQjsvibDnNngvc/BALP
XeSJuh1oO3NOCRC5rM+nozq4sTOZxZ5iEZrlKF33x6B/NfXaftV3OeCK6DTWuxZU
OBZeEAuVNh1HNmT1w0mKCfXAfF3LhWht5I90xF+3CD9yyqjqfHaAdIro7TVeAmVf
kjfeXZjZ8JgFIQJKTPHuN8s3GZVmya70h1fh2nEXgRxsMVEzBVO7dWC1NaXXtKnc
5kOFMwQM5X9YahbQrCZAwiPMS/HwNiJqUutjnnySU2MHQNerP3GujWAWR/dpMypj
RABz59o7vVLjJgl1pPBrJZOBMFE59LzOnHxC5izcesQDcMUDINWCwmcNJBCAkgUk
0CxYosf6Q1as5xYJL5sGXQ78Gzqd8EAeKna0jxGfjXA3TiD/IHyEmA0Kzd3WMTRB
ZZtbEnZQxtdqZD3UfDz3poR5D4GE6JdVwELILW8URnodJGQRhiZtDZVU+M1Bo5PL
Bxxx
=olTZ
-----END PGP MESSAGE-----


\bibliographystyle{acmtrans}
\bibliography{socnet}

\begin{received}
	Received unknown;
	accepted unknown;
\end{received}

\end{document}

% Local Variables:
% ispell-local-dictionary: "american"
% End: