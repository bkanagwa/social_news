\section{Introduction} % (fold)
\label{sec:introduction}

The recent development of social networking sites (e.g., {\tt facebook}, {\tt LinkedIn}, {\tt Twitter}, {\tt digg}) has
revived research in social network systems. The predominant approaches either follow \emph{networking modeling
techniques} (see~\cite{Berkowitz-Wellman:88}) or \emph{statistical}, and more precisely data mining, approaches for
information extraction and prediction. However, these different approaches have occulted one important advantage of
social networks. They can serve as a structure to provide a huge ``manpower'' in distributed problem solving. Lately, we
have witnessed early developments of this trend in \emph{crowd sourcing} and \emph{Google Image Labeler} based on {\tt
ESP Game} designed by \emph{Luis Von Ahn} (see~\url{http://www.espgame.org}).

Similarly to \emph{Grid Computing} offering a huge computational power, social networks can tremendously contribute to
improving decision making in \emph{complex adaptive systems}\footnote{A complex adaptive system should be understood as a
system which involves a great deal of interaction and where the entities can easily adjust to changes in their
environment.}. However, this demands a further understanding of social networks, and more particularly their behavioral
perspective. In thoroughly understanding human decision making processes in social networks, one can capture the
reproducible parts and automate them in order to support the design of concurrent systems. As well, advanced
\emph{interaction protocols} can be derived from the interactions between humans during their task sharing in a social
network. This research attempts to \emph{formally} specify and analyze collaboration (communication and concurrent
behaviors) of humans in a social network.

Various formalisms/approaches have been developed in \emph{Theoretical Computing} to specify, analyze and verify the
properties of concurrent systems (see~\cite{Winskel-Nielsen:95}). These range from \emph{automata}-based formalisms
(e.g., \emph{communicating automata} and \emph{Petri Nets}) to various process algebra (e.g., CCS, CSP, $\pi-calculus$,
etc.) All these formalisms/approaches can capture the properties common to social networks: \emph{concurrency},
\emph{mobility}, \emph{distribution} and \emph{communication}.

Given the empirical nature of our research, the applications we use to study the behaviors of human users in social
networks determine the quality and ``profoundness'' of the understanding we gain in return. We adopt a two-layer model
where the social network is considered as a ``static structure"\footnote{Even though the topology of a social network is
expected to evolve over time, we only consider its snapshot at the beginning of the application.} on top of which the
application is deployed, i.e., the application shows how a community materializes within the social network. In this
paper we discuss a social networking application, {\tt Social News}, where members of a social network adopt
\emph{headlines} from a set of news concurrently submitted by some members of the social network. The evolution of a news
to a headline follows three steps: \begin{inparaenum}[\itshape 1\upshape)] \item initial group extension; \item local
headline adoption; followed by \item global headline adoption. \end{inparaenum} We specify and analyze the communications
and behaviors of the members of the social network during the process using both $\pi-calculus$ and \emph{Petri Net}.
$\pi-calculus$ is used for the intra-group communication until the adoption of local headlines, and then \emph{Petri Net}
is used to specify and analyze the global behaviors. Especially, we use a variant of $\pi-calculus$ which caters for
group formation and confidential communication within.

The first attempts to combine the visual effects of \emph{Petri Net} and the expressivity of $\pi-calculus$ trace back to
$\pi-nets$ (\cite{Milner:94}) and \emph{interaction diagrams} (\cite{Parrow:95}). Recently, Milner introduced
\emph{BiGraphs}, a combination of locality and connectivity to reason about concurrent systems
(see~\cite{Milner:08,Milner:09}). In general, diagrammatic formalisms are appealing for a number of reasons, including
easiness of modeling, support for representation, conservation of system structure and normalization. However, when the
size of the concepts being represented grows bigger they lack the conciseness of term rewriting-based formalisms. In
combining both approaches in this research, we aim at offering the best of both worlds. Moreover, we take a particular
care of confidentiality, secrecy and group formation by adopting a more suitable variant of $\pi-calculus$.

It is noteworthy that this research does not pretend to develop a generic method to analyze communication and concurrent
behaviors in a social network. Rather, we wish to demonstrate the potential of combining several formal approaches to
further understand distributed decision making in a social network, just as in~\cite{He-Yuan-Zeng:07}. We argue that the
novelty in our approach is twofold. First, we break free from the usual statistical approaches, which can only hint at
trends but usually fall short to provide a rigorous interpretation. Second, we claim that our two-layer modeling is
unprecedented. By separating the social networking application from the social network itself, we can model the former
using any advanced logical formalism, while still regarding the later as a ``network of people''. Thus, any behavior
observed in the system is simply the impact of the social network on the type of application being studied.

The remainder of the paper is organized as follows. Section~\ref{sec:social_networks} introduces the concept of social network,
while section~\ref{sec:formalisms} discusses the formalisms we use. Section~\ref{sec:application} discusses {\tt Social News},
our social networking application. Sections~\ref{sec:group_extension_local_headline} and \ref{sec:global_headline_adoption} are the main contributions of
the paper. The former specifies the (interactions during) group formation and local headline adoption, while the later
models the entire social network once the groups have been formed and local headlines adopted. Finally,
section~\ref{sec:conlusion} draws conclusions and highlights future work.

% section introduction (end)